%D \module
%D   [      file=simpleslides-c-default,
%D        version=2010.09.20
%D          title=\CONTEXT\ Style File,
%D       subtitle=Presentation Module simpleslides --- Default counter setup,
%D         author=Aditya Mahajan and Thomas A. Schmitz,
%D           date=\currentdate,
%D      copyright={Aditya Mahajan and Thomas A. Schmitz}]
%C
%C Copyright 2010 Aditya Mahajan and Thomas A. Schmitz
%C This file may be distributed under the GNU General Public License v. 2.0.

\writestatus{simpleslides}{loading default counter setup}

\startmodule[simpleslides-c-default]

\unprotect

%D Attempt to make the counters for the simpleslides module more "modular."
%D This file is called by t-simpleslides.tex and provides the metapost code for
%D page counters which are then included in the backgrounds of the style
%D modules.

% \def\setupSlideCounter%
%  {\dosingleargument\dosetupSlideCounter}
% 
% \def\dosetupSlideCounter[#1]%
%   {\setvariables[simpleslides:counter]
%                 [#1]}
% 
% \processaction
%   [\getvariable{simpleslides:counter}{alternative}]
%   [        a=>\def\simpleslidescounter{simpleslides:shaded:counter1},
%            b=>\def\simpleslidescounter{simpleslides:shaded:counter2},
%            c=>\def\simpleslidescounter{simpleslides:shaded:counter3},
%   \v!default=>\def\simpleslidescounter{simpleslides:shaded:counter1},
%   \v!unknown=>\def\simpleslidescounter{simpleslides:shaded:counter1}]

\def\simpleslidescounter{simpleslides:shaded:counter4}

\startuseMPgraphic{simpleslides:shaded:counter1}
save p ; path p[] ;
save a ; 
if NOfPages <= 15:
	a = 0.7cm ;
elseif NOfPages <= 30:
	a = 0.45cm ;
else :
	a = 0.25cm ;
fi ;
save factor ; numeric factor ;
factor = (TextWidth - a)/(NOfPages - 2) ;
for i = 2 upto NOfPages :
	p[i] = unitcircle scaled a shifted (BackSpace + (i-2)*factor, (2cm - a)/2) ;
	if i = PageNumber :
		pickup pencircle scaled 5pt ;
		drawfill p[i] withcolor \MPcolor{simpleslides:variantcolor} ;
	else :
		circular_shade(p[i],2,\MPcolor{simpleslides:backgroundcolor}, \MPcolor{simpleslides:variantcolor}) ;
	fi
endfor ;
\stopuseMPgraphic

\startuseMPgraphic{simpleslides:shaded:counter2}
save a,b ;
numeric a,b ;
a = 7mm ;
b = PaperWidth/2 - NOfPages * 2.5pt ;

save p,q; path p,q ;
p =((0,5mm)    -- (1mm,11mm)) shifted (b,0) ;
q =((-8mm,5mm) -- (0,11mm)  ) shifted (b,0) ;

pickup pencircle scaled 3pt ;
for i := NOfPages-1 downto 1:
  draw (if i mod 5 = 0 : q else : p fi)
       shifted (i*5pt, 0pt)
       withcolor if i < PageNumber : \MPcolor{simpleslides:variantcolor} 
                 else :              \MPcolor{simpleslides:contrastcolor}
                 fi ;
endfor ;
\stopuseMPgraphic

\startuseMPgraphic{simpleslides:shaded:counter3} 
save diff ;numeric diff; 
if NOfPages <= 25:
	diff = 0.4cm ;
elseif NOfPages <= 35:
	diff = 0.3cm ;
else :
	diff = 0.2cm ;
fi ;

save factor; numeric factor; 
factor = TextWidth/(NOfPages - 2) ;

save p; path p ;
p = unitsquare xyscaled (diff,diff) 
               shifted (BackSpace,0.85cm) ;

for i = 2 upto NOfPages:
  if PageNumber = i:
    fill p xyscaled (0,2) shifted ( (i-2)*factor,-diff-0.85cm) 
         withcolor \MPcolor{simpleslides:contrastcolor} ;
  else :
    fill p shifted ( (i-2)*factor, 0) 
         withcolor \MPcolor{simpleslides:variantcolor} ;
  fi ;
endfor ;
\stopuseMPgraphic 

\startuseMPgraphic{simpleslides:shaded:counter4}
save p ;
path p[] ;
linejoin := rounded ;

p[1] := unitsquare xyscaled(TextWidth,1cm) shifted (BackSpace,0.8cm) ;
save i ;
numeric i; 
i = PageNumber/NOfPages ;

p[2] = ulcorner p[1] -- urcorner p[1] ;
p[3] = llcorner p[1] -- lrcorner p[1] ;

save o;
pair o[] ;

o[1] := point i along p[2] ;
o[2] := point i along p[3] ;

p[4] = ulcorner p[1] -- o[1] -- o[2] -- llcorner p[1] -- cycle ;

fill p[1] withcolor 0.7\MPcolor{simpleslides:backgroundcolor} ;
fill p[4] withcolor \MPcolor{simpleslides:contrastcolor} ;

pickup pencircle scaled 1pt ;
draw ulcorner p[1] -- urcorner p[1] -- lrcorner p[1] withcolor 0.3white ;
draw lrcorner p[1] -- llcorner p[1] -- ulcorner p[1] withcolor 0.8white ;

%clip currentpicture to p[4] ;

% linear_shade(p[1],0,white,red) ;
% 
% %            \MPcolor{simpleslides:backgroundcolor},
% %            \MPcolor{simpleslides:contrastcolor}) ;
% 
% 
% 
% 
\stopuseMPgraphic

\protect

\stopmodule
