%D \module
%D   [      file=simpleslides-s--Embossed,
%D        version=2009.03.30
%D          title=\CONTEXT\ Style File,
%D       subtitle=Presentation Module --- Embossed style,
%D         author=Aditya Mahajan and Thomas A. Schmitz,
%D           date=\PRESTITdate,
%D      copyright={Aditya Mahajan and Thomas A. Schmitz}]
%C
%C Copyright 2007 Aditya Mahajan and Thomas A. Schmitz
%C This file may be distributed under the GNU General Public License v. 2.0.

%D This file provides the \quotation{embossed} style for the presentation
%D module. It is loaded at runtime. 

\writestatus{simpleslides}{loading module embossed}

\startmodule[simpleslides-s-Embossed]

\unprotect
%D The page layout:

\setuplayout [width=fit,
              margin=1.3cm,
              height=fit,
              header=1cm, 
              footer=1cm, 
              topspace=10mm, 
              backspace=2cm,
              location=singlesided]

\setuplayout [simpleslides:layout:horizontal][header=1cm]
\setuplayout [simpleslides:layout:vertical]  [header=0cm]
\setuplayout [simpleslides:layout:title]     [header=0cm]

%D These macros are used for placing figures/pictures:

\define\NormalHeight{.94\textheight}
\define\NormalWidth{.476\textwidth}
\define\PictureFrameHeight{.94\textheight}
\define\PictureFrameWidth{.476\textwidth}

%D We also specify the position of the slidetitle.

\setuplayer[simpleslides:layer:slidetitle]
    [x=20mm,y=2mm]

%D Next we define a generic frame

\defineframed[simpleslides:framed]
             [frame=off,offset=0pt,
              top=\vss,bottom=\vss]

\defineframed[simpleslides:framed:big]
             [frame=off,offset=0pt,strut=no,
              width=\textwidth,height=2cm,
              top=\vss,bottom=\vss]

%D We define our color scheme:

\definecolor [simpleslides:backgroundcolor]   [r=1,g=1,b=.8]
\definecolor [simpleslides:variantcolor]    [r=.6,g=.2,b=.2]
\definecolor [simpleslides:specialcolor:1] 	[r=.4,g=.2,b=.2]
\definecolor [simpleslides:specialcolor:2] 	[r=.7,g=.2,b=.2]
\definecolor [simpleslides:contrastcolor] 	[r=.2,g=.2,b=.5]
\definecolor [simpleslides:framecolor]      [s=.4]
\definecolor [simpleslides:itemize:color]     [simpleslides:contrastcolor]

%D This module write "Made with ConTeXt" in bold at the bottom. We first define
%D a font for it.

\beginOLDTEX
\loadmapfile[qhv-ec.map]
\definefontsynonym [Embossed] [ec-qhvb] 
\endOLDTEX

\beginXETEX
\loadmapfile[qhv-ec.map]
\definefontsynonym [Embossed] [ec-qhvb] 
\endXETEX

\beginLUATEX
\definefontsynonym [Embossed] [name:texgyreherosbold]
\endLUATEX

\definefont[EmblemFont] [Embossed at 30pt]

%D The emblem string is configurable.

\setuplabeltext [\s!en]    [simpleslidesemblem={Made with \CONTEXT}]

\definetextext[simpleslides:sometxt:left] {\TaspresentSometxtLeft}
\definetextext[simpleslides:sometxt:right]{\TaspresentSometxtRight}

\unexpanded\def\TaspresentSometxtLeft#1%
  {\getvalue{simpleslides:framed:big}
                       {\EmblemFont\color[simpleslides:specialcolor:1]
                       {#1}}}

\unexpanded\def\TaspresentSometxtRight#1%
  {\getvalue{simpleslides:framed:big}
                       {\EmblemFont\color[simpleslides:specialcolor:2]
                       {#1}}}

\startuseMPgraphic{simpleslides:MP:ornament}
StartPage ;

save a, b;
numeric a; a = 2cm ;
numeric b; b = 0.9cm ;

z1 = llcorner Page shifted (0,a) ;
z2 = lrcorner Page shifted (0,a) ;

save bottom ; path bottom ;
bottom = llcorner Page -- z1 -- z2 -- lrcorner Page -- cycle ;

fill Page withcolor \MPcolor{simpleslides:backgroundcolor} ;
fill bottom withcolor \MPcolor{simpleslides:variantcolor} ;

draw Page withcolor \MPcolor{simpleslides:framecolor} 
          withpen pencircle scaled 12pt ;
draw topboundary bottom withcolor \MPcolor{simpleslides:framecolor}
          withpen pencircle scaled 6pt ;


draw \sometxt[simpleslides:sometxt:left]{\labeltext{simpleslidesemblem} \hfill \pagenumber\ of \lastpage} 
      shifted (1.96cm,0.04cm) ;

draw \sometxt[simpleslides:sometxt:right]{\labeltext{simpleslidesemblem} \hfill \pagenumber\ of \lastpage} 
      shifted (2cm,0) ; 

StopPage ;
\stopuseMPgraphic


%D We define these backgrounds as overlays:

\defineoverlay
  [simpleslides:background:ornament]
  [\useMPgraphic{simpleslides:MP:ornament}]

\defineoverlay
  [simpleslides:background:title]
  [\useMPgraphic{simpleslides:MP:ornament}]


%D We want the presentation title to be in color

\setupTitle[\c!title\c!color=simpleslides:contrastcolor]

%D We also want the slide title in a framed box.

\setupSlideTitle
  [\c!after=,
   \c!alternative=layer,
   \c!color={simpleslides:contrastcolor},
   \c!width=\textwidth,
   \c!height=2cm,
   \c!align=\v!middle]


%D The symbol for the first level of itemizations. 

\definesymbol[1][\useMPgraphic{simpleslides:itemize:square}]
\setupitemize[1][color={simpleslides:itemize:color}]

\protect
\stopmodule

\endinput

