%D \module
%D   [      file=t-simpleslides,
%D        version=2009.03.30
%D          title=\CONTEXT\ Style File,
%D       subtitle=Presentation Module simpleslides,
%D         author=Thomas A. Schmitz,
%D           date=\currentdate,
%D      copyright={Thomas A. Schmitz}]
%C
%C Copyright 2007 Thomas A. Schmitz.
%C This file may be distributed under the GNU General Public License v. 2.0.

%M \usemodule[int-load]
%M \setupcolors[state=start]
%M \loadsetups[cont-en.xml]
%M \loadsetups[t-simpleslides.xml]

%D This module is meant to facilitate writing presentations in \CONTEXT. It
%D provides a consistent interface and macros; there are different styles which
%D give different output. The module has been written for projector||based
%D presentations, so elements which are typical for screen presentations (such
%D as interactive hyperlinks or tables of contents) are not included. The
%D module is meant for an academic environment, specifically in the humanities.
%D Hence, it has the following characteristics:
%D
%D \startitemize 
%D
%D \item The look is rather sober. In academia, presentations are not meant to
%D showcase fancy \TeX\ effects; nothing should divert the audience's attention
%D from the content.
%D
%D \item The module is written for slides which exhibit text and/or images.
%D From my own experience with \TeX||based presentations, I have provided a
%D setup for horizontal (landscape) pictures and for vertical (portrait)
%D pictures, which are accompanied by an area for explanatory text.
%D
%D \item A simple switch in the module setup command will produce different
%D output. 
%D
%D \item It is easy to customize the module or to add more styles.
%D
%D \stopitemize
%D
%D The macros are commented rather extensively to give users (especially users
%D relatively new to \CONTEXT) the chance to understand the mechanisms and
%D create their own styles. Of course, I did not invent this code on my own. My
%D thanks are due, as always, to Hans Hagen, whose presentation modules in the
%D \CONTEXT\ core have been a wonderful source of inspiration, to Mojca
%D Miklavec, who provided help with Metapost, and to Aditya Mahajan, who helped
%D tremendously in cleaning up the code and making the user interface more
%D consistent. 

\writestatus{loading}{module simpleslides}

\startmodule[simpleslides]

\unprotect

\setupmodule
  [style=DoubleFrame,
   font=LatinModernSans,
   size=17pt,
   color=blue,
   bottom=square]

\usemodule
  [simpleslides-s]
  [default]

\doiffileelse{\currentmoduleparameter{style}}
  {\usemodule{\currentmoduleparameter{style}}}
  {\usemodule[simpleslides-s][\currentmoduleparameter{style}]}
  [ color=\currentmoduleparameter{color},
   bottom=\currentmoduleparameter{bottom}]

\usemodule
  [simpleslides-f]
  [default]

\protect

\stopmodule

% Demo in the end.
