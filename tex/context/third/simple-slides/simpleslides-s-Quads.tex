%D \module
%D   [      file=simpleslides-s-Quads,
%D        version=2007.07.17, 
%D          title=\CONTEXT\ Style File,
%D       subtitle=Presentation Module Quads,
%D         author=Thomas A. Schmitz,
%D           date=\PRESTITdate,
%D      copyright={Thomas A. Schmitz}]
%C
%C Copyright 2007 Thomas A. Schmitz.
%C This file may be distributed under the GNU General Public License v. 2.0.

%D This file provides the \quotation{Quads} style for the presentation
%D module. It is loaded at runtime. 

\writestatus{simpleslides}{loading Quads style}

\startmodule[simpleslides-s-Quads]

\unprotect

%D First, we define the page layout.

\setuplayout [width=fit,
	      height=middle,
              margin=1.5cm,
              height=fit,
	      leftmargindistance=.4cm,
	      rightmargindistance=0cm,
              header=1.5cm, 
              footer=0cm, 
              topspace=1cm, 
              backspace=2.5cm,
	      cutspace=1.5cm,
              location=singlesided]

\setuplayout [simpleslides:layout:horizontal][header=15mm]
\setuplayout [simpleslides:layout:vertical]  [header=0mm]
\setuplayout [simpleslides:layout:title]     [header=0mm]

%D These macros are used for placing figures/pictures:

\define\NormalHeight        {\textheight}
\define\NormalWidth         {.5\textwidth}
\define\PictureFrameHeight  {\textheight}
\define\PictureFrameWidth   {.5\textwidth}

\setuplayer
  [simpleslides:layer:slidetitle]
    [x=25mm,
     y=2mm]

%D We define our color scheme:

\definecolor[simpleslides:contrastcolor]	[r=0,g=0,b=.92]
\definecolor[simpleslides:altcontrastcolor]	[r=0,g=0,b=.4]
\definecolor[simpleslides:backgroundcolor]	[s=.98]
\definecolor[simpleslides:itemize:color]	[r=0,g=0,b=.92]

%D We let Metapost calculate the background:

\startuseMPgraphic{simpleslides:MP:horizontal} 
StartPage ;
fill Page withcolor \MPcolor{simpleslides:backgroundcolor} ;
z1 = ulcorner Page ;
z5 = llcorner Page ;
path q ;
q = z1 -- z5 ;
t := arclength (q) ;
u := t/15 ;
v := (PageNumber/NOfPages) ;
z4 = (x1+1cm, y1-1cm) ;
z3 = (x4, y1) ;
z2 = (x1, y4) ;
path m[] ;
m[1] = z1 -- z2 -- z4 -- z3 -- cycle ;
m[2] = m[1] shifted (0, -2*u) ;
m[3] = m[1] shifted (0, -4*u) ;
m[4] = m[1] shifted (0, -6*u) ;
m[5] = m[1] shifted (0, -8*u) ;
m[6] = m[1] shifted (0, -10*u) ;
m[7] = m[1] shifted (0, -12*u) ;
m[8] = m[1] shifted (0, (-14*u-0.5mm)) ;
for i=1 upto 8:
	fill m[i] withcolor\MPcolor{simpleslides:contrastcolor} ;
endfor;
if PageNumber=1:
	fill m[1] withcolor \MPcolor{simpleslides:altcontrastcolor} ;
elseif (v>.001) and (v<.167) :
	fill m[2] withcolor \MPcolor{simpleslides:altcontrastcolor} ;	
elseif (v>.166) and (v<.334):
	fill m[3] withcolor \MPcolor{simpleslides:altcontrastcolor} ;	
elseif (v>.333) and (v<.501):
	fill m[4] withcolor \MPcolor{simpleslides:altcontrastcolor} ;	
elseif (v>.5) and (v<.667):
	fill m[5] withcolor \MPcolor{simpleslides:altcontrastcolor} ;	
elseif (v>.666) and (v<.834):
	fill m[6] withcolor \MPcolor{simpleslides:altcontrastcolor} ;	
elseif (v>.833) and (v<1):
	fill m[7] withcolor \MPcolor{simpleslides:altcontrastcolor} ;	
elseif v=1:
	fill m[8] withcolor \MPcolor{simpleslides:altcontrastcolor} ;
fi ;
StopPage ;
\stopuseMPgraphic 

%D We define these backgrounds as overlays:

\defineoverlay 
  [simpleslides:background:horizontal] 
  [\useMPgraphic{simpleslides:MP:horizontal}] 

\defineoverlay 
  [simpleslides:background:vertical] 
  [\useMPgraphic{simpleslides:MP:horizontal}] 

\defineoverlay 
  [simpleslides:background:title] 
  [\useMPgraphic{simpleslides:MP:horizontal}] 

\defineoverlay 
  [simpleslides:background:ornament] 
  [\useMPgraphic{simpleslides:MP:horizontal}] 

%D We want the title to placed in color. 

\setupTitle[\c!headcolor={simpleslides:contrastcolor}]

%D This sets up the SlideTitle:

\setupSlideTitle
   [\c!after=,
    \c!alternative=layer,
    \c!width=\textwidth,
    \c!align=\v!center,
    \c!height=2.5cm,
    \c!color=simpleslides:contrastcolor]

%D We set up the titlepage:

\setupTitle
  [\c!title\c!color={simpleslides:contrastcolor},
   \c!author\c!color={simpleslides:contrastcolor},
   \c!date\c!color={simpleslides:contrastcolor}]

\setupcombinations[distance=0cm]

%D The symbol for the first level of itemizations. 

\definesymbol[1][$\square$]
\setupitemize[1][inmargin][color=simpleslides:itemize:color]

\protect
\stopmodule

\endinput

