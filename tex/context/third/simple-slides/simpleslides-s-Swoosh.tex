%D \module
%D   [      file=simpleslides-s-Swoosh,
%D        version=2009.02.04, 
%D          title=\CONTEXT\ Style File,
%D       subtitle=Presentation Module -- Swoosh Style,
%D         author=Thomas A. Schmitz,
%D           date=\currentdate,
%D      copyright={Thomas A. Schmitz}]
%C
%C Copyright 2009 Thomas A. Schmitz.
%C This file may be distributed under the GNU General Public License v. 2.0.

%D This file provides the \quotation{Swoosh} style for the presentation
%D module. It is loaded at runtime.

\writestatus{simpleslides}{loading Swoosh style}

\startmodule[simpleslides-s-Swoosh]

\unprotect

%D First we change the page layout, adding more space on the top.

\setuplayout [width=fit,
              margin=0cm,
              height=fit,
              header=3cm, 
              footer=0.8cm, 
              topspace=.6cm, 
              backspace=1cm,
              location=singlesided]

\setuplayout [simpleslides:layout:horizontal][header=3cm]
\setuplayout [simpleslides:layout:vertical]  [header=0.8cm]
\setuplayout [simpleslides:layout:title]     [header=0.8cm]

%D We also specify the position of the slidetitle.

\setuplayer[simpleslides:layer:slidetitle]
    [x=10mm,y=2mm]

\setupcombinations[distance=1.85cm]

%D These macros are used for placing figures/pictures:

\define\NormalHeight        {\textheight}
\define\NormalWidth         {.45\textwidth}
\define\PictureFrameHeight  {\textheight}
\define\PictureFrameWidth   {.45\textwidth}

\defineframed[simpleslides:framed]
             [frame=off,offset=0pt,
              top=\vss,bottom=\vss]

%D We define our color scheme:

\definecolor [simpleslides:backgroundcolor]     [r=.88,g=.92,b=.95]
\definecolor [simpleslides:contrastcolor]	[r=.4,g=.6,b=.8]
\definecolor [simpleslides:altcontrastcolor]	[r=.1,g=.1,b=.4]
\definecolor [simpleslides:itemize:color]       [simpleslides:contrastcolor]
\definecolor [simpleslides:textcolor]  		[simpleslides:altcontrastcolor] 

\setupcolors[textcolor=simpleslides:textcolor]

%D We use \METAPOST\ to draw backgrounds.

\startuniqueMPgraphic{simpleslides:MP:horizontal} 
StartPage ;
save a, b;
numeric a ; a = 2.5cm ;
numeric b ; b = 0.7cm ;
fill Page withcolor \MPcolor{simpleslides:backgroundcolor} ;

z1 = ulcorner Page ;
z2 = urcorner Page ;
z3 = (x1,y1-0.7*a) ;
z4 = (x1+5cm,y1-a-b) ;
z5 = (x1+12cm,y1-a) ;
z6 = (x1+14cm,y1-a+0.3cm) ;
z7 = (x2,y2-a-b) ;

save p ;
path p ;
p = z1--z3..z4..z5..z6..z7--z2--cycle ;
fill p withcolor \MPcolor{simpleslides:contrastcolor} ;
pickup pencircle scaled 4pt ;
draw z3..z4..z5..z6..z7 withcolor white ;
StopPage ;
\stopuniqueMPgraphic 

\startuniqueMPgraphic{simpleslides:MP:vertical} 
StartPage ;

save a,b;
numeric a ; a = 2cm ;
numeric b ; b = 1cm ;
pair  t[] ;

fill Page withcolor \MPcolor{simpleslides:backgroundcolor} ;
z1 = ulcorner Page ;
z2 = llcorner Page ;
z3 = center Page ;
z4 = (x3,y1) ;
z5 = (x3+b,y1-4cm) ;
z6 = (x3+b/2,y1-7cm) ;
z7 = (x3-b/2,y2+2cm) ;
z8 = (x3,y2) ;

save p;
path p[] ;
p[1] = z1--z4..z5..z6..z7..z8--z2--cycle ;
fill p[1] withcolor \MPcolor{simpleslides:contrastcolor} ;
pickup pencircle scaled 4pt ;
draw z4..z5..z6..z7..z8 withcolor white ;
StopPage ;
\stopuniqueMPgraphic 

\startuseMPgraphic{simpleslides:MP:ornament}
save b, s, t, p, circcenter, theta, pic ;
path p[] ;
pair t[] ;
pair s[] ;
pair circcenter ; circcenter = lrcorner Page shifted (-1cm, 1cm) ;
picture pic ;
b = 1.5cm ;
StartPage ;
if \realfolio > 1:
	theta = (PageNumber - 1)/(NOfPages - 1) ;
	p[4] = fullcircle scaled b rotated 90 ;
	fill p[4] withcolor  \MPcolor{simpleslides:altcontrastcolor} ;
	t[0] = center p[4] ;
	t[1] = point 1 along p[4] ;
	t[2] = point -theta along p[4] ;
	t[3] = point -theta/2 along p[4] ;
	p[5] = t[0] -- t[1] .. t[3] .. t[2] -- cycle ;
	fill p[5] withcolor \MPcolor{simpleslides:contrastcolor} ;
	for i = 1 upto NOfPages :
		s[i] = point i/(NOfPages -1) along p[4] ;
		pickup pencircle scaled 1pt ;
		draw s[i] -- t[0] withcolor \MPcolor{simpleslides:backgroundcolor} ;
	endfor ;
	p[3] = p[4] scaled 0.4 ;
	fill p[3] withcolor \MPcolor{simpleslides:contrastcolor} ;
	draw p[3] withcolor \MPcolor{simpleslides:backgroundcolor} ;
	label(textext("\switchtobodyfont[10pt]\color[simpleslides:backgroundcolor]\pagenumber"),origin) ;
	currentpicture := currentpicture shifted circcenter ;
fi ;
StopPage ;
\stopuseMPgraphic 

%D We define these backgrounds as overlays:

\defineoverlay 
  [simpleslides:background:title] 
  [\useMPgraphic{simpleslides:MP:horizontal}] 

\defineoverlay 
  [simpleslides:background:horizontal] 
  [\useMPgraphic{simpleslides:MP:horizontal}] 

\defineoverlay 
  [simpleslides:background:vertical] 
  [\useMPgraphic{simpleslides:MP:vertical}] 

\defineoverlay
  [simpleslides:background:ornament]
  [\useMPgraphic{simpleslides:MP:ornament}]

\setupTitle
  [\c!headcolor={simpleslides:altcontrastcolor}]

%D We want the title to placed in a framed box. We redefine all the keys of
%D \type{\setupTitle}, so that the module is easier to maintain.

\setupSlideTitle
  [\c!color=simpleslides:backgroundcolor,
   \c!alternative=layer,
   \c!align=\v!center,
   \c!width=\textwidth,
   \c!height=2cm,
   \c!after=]


% \setupTitle
%   [\c!title=,
%    \c!author=,
%    \c!date=\currentdate,
%    \c!headstyle=,
%    \c!headcolor=,
%    \c!align=\v!middle,
%    \c!before={\vfill\getvalue{simpleslides:framed}
%               [\c!width=\textwidth,\c!height=.75\textheight,
%               \c!align=\v!middle, \c!strut=\v!no]
%               \bgroup},
%    \c!after={\egroup\vfill},
%    \c!title\c!style={\switchtobodyfont[\TitleSize]},
%    \c!title\c!color=,
%    \c!title\c!align=,%\v!middle,
%    \c!author\c!style=,
%    \c!author\c!color=,
%    \c!author\c!align=,%\v!middle, 
%    \c!date\c!style=,
%    \c!date\c!color=,
%    \c!date\c!align=,%\v!middle,
%    \c!before\c!title=,
%    \c!before\c!author=,
%    \c!before\c!date=,
%    \c!after\c!title={\blank[1*line]},
%    \c!after\c!author={\blank[2*line]},
%    \c!after\c!date=]
% 
% %D We want the title to be of a specific height
% 
% \setuphead[SlideTitle]
%    [\c!after=,
%     \c!alternative=\v!text,
%     \c!color=white,
%     \c!command=\doSlideTitle]
% 
% \define[2]\doSlideTitle
%   {\setlayer[simpleslides:layer:slidetitle]%
%      {\getvalue{simpleslides:framed}[\c!width=\textwidth,\c!height=1.1cm,
%                 \c!align=\v!right]
%                {#1#2}}} 

%D The symbol for the first level of itemizations. 

\definesymbol[1][\useMPgraphic{simpleslides:itemize:square}]
\setupitemize[1][color=simpleslides:itemize:color]

\protect
\stopmodule

\endinput

