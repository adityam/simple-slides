%D \module
%D   [      file=simpleslides-s-Sunrise,
%D        version=2009.03.30
%D          title=\CONTEXT\ Style File,
%D       subtitle=Presentation Module Sunrise
%D         author=Thomas A. Schmitz,
%D           date=\currentdate,
%D      copyright={Thomas A. Schmitz}]
%C
%C Copyright 2007 Thomas A. Schmitz.
%C This file may be distributed under the GNU General Public License v. 2.0.

%D This file provides the \quotation{Sunrise} style for the presentation
%D module. It is loaded at runtime. The look of this style was inspired by the
%D \quotation{husky} theme of the \LaTeX\ {\tt powerdot} package, created by
%D Jack Stalnaker.

\writestatus{simpleslides}{loading Sunrise style}

\startmodule[simpleslides-s-Sunrise]

\unprotect

%D The page layout:

\setuplayout [width=fit,
              margin=0cm,
              height=fit,
              header=2.75cm, 
              footer=1.5cm, 
	      footerdistance=0.4cm,
              topspace=0cm, 
              backspace=1cm,
              location=singlesided]

\setuplayout [simpleslides:layout:horizontal][header=2.75cm]
\setuplayout [simpleslides:layout:vertical]  [header=0.4cm]
\setuplayout [simpleslides:layout:title]     [header=2.75cm]

\setuplayer
  [simpleslides:layer:slidetitle]
  [x=10mm]

%D These macros are used for placing figures/pictures:

\define\NormalHeight        {\textheight}
\define\NormalWidth         {.476\textwidth}
\define\PictureFrameHeight  {\textheight}
\define\PictureFrameWidth   {.476\textwidth}

%D We define our color scheme:

\definecolor [simpleslides:variantcolor]      [s=.97]
\definecolor [simpleslides:backgroundcolor]   [s=.88]
\definecolor [simpleslides:contrastcolor]     [r=.75]
\definecolor [simpleslides:itemize:color]     [simpleslides:contrastcolor]

%D We let \METAPOST\ calculate the background:

%D Both horizontal and vertical group share this part of the background. 

\startuseMPgraphic{simpleslides:MP:common} 
save a, b ;
numeric a; a=2.1cm ;
numeric b; b=1.5cm ;

fill Page withcolor \MPcolor{simpleslides:variantcolor} ;

z1 = llcorner Page shifted (0,2*a) ;
z2 = z1 shifted (0,2*a) ;
z3 = lrcorner Page shifted (0,b) ;
z4 = z3 shifted (0,b) ;
z5 = z2 shifted (0,b) ;
z6 = ulcorner Page  shifted (.1cm,0) ;
z7 = z4 shifted (0,b/2) ;
z8 = z7 shifted (0,b) ;
z9 = ulcorner Page shifted (.1cm+a,0) ;
z10 = z9 shifted (3*a,0) ;
z11 = z8 shifted (0,b/2) ;
z12 = z11 shifted (0,b) ;
z13 = z10 shifted (a,0) ;
z14 = z13 shifted (3*a,0) ;
z15 = z12 shifted (0,b/2) ;
z16 = z15 shifted (0,b) ;
z17 = llcorner Page shifted (0,b) ;

save p ;
path p[] ;
p[1] = z1 -- z2 -- z4 -- z3 -- cycle ;
p[2] = z5 -- ulcorner Page -- z6 -- z8 -- z7 -- cycle ;
p[3] = z9 -- z10 -- z12 -- z11 -- cycle ;
p[4] = z13 -- z14 -- z16 -- z15 -- cycle ;
p[5] = llcorner Page -- z17 -- z3 -- lrcorner Page -- cycle ;

fill p[1] withcolor \MPcolor{simpleslides:backgroundcolor} ;
fill p[2] withcolor \MPcolor{simpleslides:backgroundcolor} ;
fill p[3] withcolor \MPcolor{simpleslides:backgroundcolor} ;
fill p[4] withcolor \MPcolor{simpleslides:backgroundcolor} ;
fill p[5] withcolor \MPcolor{simpleslides:contrastcolor} ;
\stopuseMPgraphic 

\startuniqueMPgraphic{simpleslides:MP:vertical}
StartPage ;
\includeMPgraphic{simpleslides:MP:common} ;
StopPage ;
\stopuniqueMPgraphic

\startuniqueMPgraphic{simpleslides:MP:horizontal}
StartPage ;
\includeMPgraphic{simpleslides:MP:common} ;
z18 = ulcorner Page shifted (0,-1.5*b) ;
z19 = z18 shifted (0,-1pt) ;
z20 = urcorner Page shifted (0,-1.5*b) ;
z21 = z20 shifted (0,-1pt) ;

p[6] = z18 -- z19 -- z21 -- z20 -- cycle ;

linear_shade(p[6],0,
             \MPcolor{simpleslides:contrastcolor},
             \MPcolor{simpleslides:variantcolor}) ;

p[7] = p[6] shifted (0,-3pt) ;

linear_shade(p[7],0,
             \MPcolor{simpleslides:contrastcolor},
             \MPcolor{simpleslides:variantcolor}) ;

StopPage ;
\stopuniqueMPgraphic

%D We define these backgrounds as overlays:

\defineoverlay 
  [simpleslides:background:title] 
  [\useMPgraphic{simpleslides:MP:horizontal}] 

\defineoverlay 
  [simpleslides:background:horizontal] 
  [\useMPgraphic{simpleslides:MP:horizontal}] 

\defineoverlay 
  [simpleslides:background:vertical] 
  [\useMPgraphic{simpleslides:MP:vertical}] 

%D We define the footer

\setupfooter[\c!color=simpleslides:variantcolor,
             \c!style={\switchtobodyfont[10pt]},
             \c!strut=\v!yes]

\setupfootertexts[{\framed[\c!frame=\v!off,
                           \c!height=1cm,
                           \c!width=\textwidth]
                           {\simpleslidestitleparameter{title}
                            \hfill \pagenumber\ of \lastpage}}]

%D We want the title to placed in a framed box. We redefine all the keys of
%D \type{\setupTitle}, so that the module is easier to maintain.

\setupTitle
  [\c!title=,
   \c!author=,
   \c!date=\currentdate,
   \c!headstyle=,
   \c!headcolor=simpleslides:contrastcolor,
   \c!align=\v!middle,
   \c!before={\vfill\getvalue{simpleslides:framed}
              [\c!width=\textwidth,\c!height=.75\textheight,
               \c!align=\v!middle, \c!strut=\v!no]
              \bgroup},
   \c!after={\egroup\vfill},
   \c!title\c!style={\switchtobodyfont[\TitleSize]},
   \c!title\c!color=,
   \c!title\c!align=,%\v!middle,
   \c!author\c!style=,
   \c!author\c!color=,
   \c!author\c!align=,%\v!middle, 
   \c!date\c!style=,
   \c!date\c!color=,
   \c!date\c!align=,%\v!middle,
   \c!before\c!title=,
   \c!before\c!author=,
   \c!before\c!date=,
   \c!after\c!title={\blank[1*line]},
   \c!after\c!author={\blank[2*line]},
   \c!after\c!date=]

\setupSlideTitle
  [\c!after=,
   \c!alternative=layer,
   \c!width=\textwidth,
   \c!height=2.25cm,
   \c!color={simpleslides:contrastcolor}]


%D The symbol for the first level of itemizations. 


\definesymbol[1][\useMPgraphic{simpleslides:itemize:square}]
\setupitemize[1][color={simpleslides:itemize:color}]

\protect
\stopmodule

\endinput

