%D \module
%D   [      file=simpleslides-s-Sunrise,
%D        version=2007.07.17, 
%D          title=\CONTEXT\ Style File,
%D       subtitle=Presentation Module Sunrise
%D         author=Thomas A. Schmitz,
%D           date=\currentdate,
%D      copyright={Thomas A. Schmitz}]
%C
%C Copyright 2007 Thomas A. Schmitz.
%C This file may be distributed under the GNU General Public License v. 2.0.

%D This file provides the \quotation{Sunrise} style for the presentation
%D module. It is loaded at runtime. The look of this style was inspired by the
%D \quotation{husky} theme of the \LaTeX\ {\tt powerdot} package, created by
%D Jack Stalnaker.

\writestatus{simpleslides}{loading Sunrise style}

\startmodule[simpleslides-s-Sunrise]

\unprotect

%D The page layout:

\setuplayout [width=fit,
              margin=0cm,
              height=fit,
              header=2.75cm, 
              footer=1.5cm, 
	      footerdistance=0.4cm,
              topspace=0cm, 
              backspace=1cm,
              location=singlesided]

\setuplayout [simpleslides:layout:horizontal][header=2.75cm]
\setuplayout [simpleslides:layout:vertical]  [header=0.4cm]
\setuplayout [simpleslides:layout:title]     [header=2.75cm]

%D In this style, we don't want the ornament background for vertical slides:

\startsetups simpleslides:background:vertical
  \setuplayout[simpleslides:layout:vertical]
  \setupbackgrounds[\v!page]
        [background={simpleslides:background:vertical}]
\stopsetups

%D We also specify the position of the slidetitle.

\setuplayer[simpleslides:layer:slidetitle]
    [width=\paperwidth,
    height=\paperheight,
    x=10mm]

%D These macros are used for placing figures/pictures:

\define\NormalHeight       {\textheight}
\define\NormalWidth        {.476\textwidth}
\define\PictureFrameHeight {\textheight}
\define\PictureFrameWidth  {.476\textwidth}

%D We define our colors:

\definecolor [simpleslides:backgroundcolor]     [s=.97]
\definecolor [simpleslides:variantcolor]        [s=.88]
\definecolor [simpleslides:contrastcolor]       [r=.75]
\definecolor [simpleslides:itemize:color]       [simpleslides:contrastcolor]

%D We let Metapost calculate the background:

\startuniqueMPgraphic{simpleslides:MP:horizontal} 
StartPage ;
pair zd[] ;
path pb[] ;
numeric a; a=2.1cm ;
numeric b; b=1.5cm ;
fill Page withcolor \MPcolor{simpleslides:backgroundcolor} ;
z.d1 = llcorner Page shifted (0,2*a) ;
z.d2 = z.d1 shifted (0,2*a) ;
z.d3 = lrcorner Page shifted (0,b) ;
z.d4 = z.d3 shifted (0,b) ;
z.d5 = z.d2 shifted (0,b) ;
z.d6 = ulcorner Page  shifted (.1cm,0) ;
z.d7 = z.d4 shifted (0,b/2) ;
z.d8 = z.d7 shifted (0,b) ;
z.d9 = ulcorner Page shifted (.1cm+a,0) ;
z.d10 = z.d9 shifted (3*a,0) ;
z.d11 = z.d8 shifted (0,b/2) ;
z.d12 = z.d11 shifted (0,b) ;
z.d13 = z.d10 shifted (a,0) ;
z.d14 = z.d13 shifted (3*a,0) ;
z.d15 = z.d12 shifted (0,b/2) ;
z.d16 = z.d15 shifted (0,b) ;
z.d17 = llcorner Page shifted (0,b) ;
pb[1] = z.d1 -- z.d2 -- z.d4 -- z.d3 -- cycle ;
fill pb[1] withcolor \MPcolor{simpleslides:variantcolor} ;
pb[2] = z.d5 -- ulcorner Page -- z.d6 -- z.d8 -- z.d7 -- cycle ;
fill pb[2] withcolor \MPcolor{simpleslides:variantcolor} ;
pb[3] = z.d9 -- z.d10 -- z.d12 -- z.d11 -- cycle ;
fill pb[3] withcolor \MPcolor{simpleslides:variantcolor} ;
pb[4] = z.d13 -- z.d14 -- z.d16 -- z.d15 -- cycle ;
fill pb[4] withcolor \MPcolor{simpleslides:variantcolor} ;
pb[5] = llcorner Page -- z.d17 -- z.d3 -- lrcorner Page -- cycle ;
fill pb[5] withcolor \MPcolor{simpleslides:contrastcolor} ;
StopPage ;
\stopuniqueMPgraphic 

\startuniqueMPgraphic{simpleslides:MP:ornament}
StartPage ;
path pb[] ;
numeric b; b=1.5cm ;
z.d18 = ulcorner Page shifted (0,-1.5*b) ;
z.d19 = z.d18 shifted (0,-1pt) ;
z.d20 = urcorner Page shifted (0,-1.5*b) ;
z.d21 = z.d20 shifted (0,-1pt) ;
pb[6] = z.d18 -- z.d19 -- z.d21 -- z.d20 -- cycle ;
linear_shade(pb[6],0,\MPcolor{simpleslides:contrastcolor},\MPcolor{simpleslides:backgroundcolor}) ;
pb[7] = pb[6] shifted (0,-3pt) ;
linear_shade(pb[7],0,\MPcolor{simpleslides:contrastcolor},\MPcolor{simpleslides:backgroundcolor}) ;
StopPage ;
\stopuniqueMPgraphic

%D We define these backgrounds as overlays:

\defineoverlay 
   [simpleslides:background:horizontal] 
   [\useMPgraphic{simpleslides:MP:horizontal}] 
 
\defineoverlay 
  [simpleslides:background:vertical] 
  [\useMPgraphic{simpleslides:MP:horizontal}] 

\defineoverlay 
  [simpleslides:background:title] 
  [\useMPgraphic{simpleslides:MP:horizontal}] 

\defineoverlay 
  [simpleslides:background:ornament] 
  [\useMPgraphic{simpleslides:MP:ornament}] 

%D this sets up the title page:

\setupTitle
  [\c!headcolor={simpleslides:contrastcolor}]

%D The slide title is typeset in a layer

\setupSlideTitle
  [\c!color={simpleslides:contrastcolor},
   \c!alternative=layer,
   \c!style={\switchtobodyfont[\TitleSize]\bf},
   \c!align=\v!center,
   \c!width=\textwidth,
   \c!height=2.25cm,
   \c!after=]

%D We define the footer

\setupfooter[color=simpleslides:backgroundcolor,style={\switchtobodyfont[10pt]},strut=no]
\setupfootertexts[{\framed[frame=off,strut=no,offset=\zeropoint,height=1.5cm,width=\textwidth]{\moduleparameter{simpleslides:title}{title}\hfill \pagenumber\ of \lastpage}}]

%D The symbol for the first level of itemizations. 

\definesymbol[1][\useMPgraphic{simpleslides:itemize:square}]
\setupitemize[1][color={simpleslides:itemize:color}]

\protect
\stopmodule

\endinput

