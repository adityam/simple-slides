%D \module
%D   [      file=simpleslides-t-SuperEllipse,
%D        version=2007.12.20, 
%D          title=\CONTEXT\ Style File,
%D       subtitle=Presentation Module quadblue,
%D         author=Thomas A. Schmitz,
%D           date=\PRESTITdate,
%D      copyright={Thomas A. Schmitz}]
%C
%C Copyright 2007 Thomas A. Schmitz.
%C This file may be distributed under the GNU General Public License v. 2.0.

%D This file provides the \quotation{SuperEllipse} style for the presentation
%D module. The design is inspired by Hans's "funny" presentation module
%D (s-pre-03). It is loaded at runtime. 

\writestatus{simpleslides}{loading SuperEllipse style}

\startmodule[simpleslides-s-SuperEllipse]

\unprotect

%D First, we change the page layout.

\setuplayout [width=fit,
              height=fit,
	      margin=0cm,
              header=1.2cm, 
              footer=0cm, 
              topspace=1.8cm, 
              backspace=1.5cm,
              location=singlesided]

%D These macros are used for placing figures/pictures:

\define\NormalHeight{\textheight}
\define\NormalWidth{.5\textwidth}
\define\PictureFrameHeight{\textheight}
\define\PictureFrameWidth{.5\textwidth}

\setuplayer
  [simpleslides:slidetitle]
    [y=10mm,
     x=15mm]

%D We define our color scheme:

\definecolor[simpleslides:outercolor]                [s=0]
\definecolor[simpleslides:backgroundcolor]           [s=.8]
\definecolor[simpleslides:contrastcolor]             [r=.5,g=0,b=0]
\definecolor[simpleslides:altcontrastcolor][r=.9,g=0,b=0]
\definecolor[simpleslides:itemize:color]             [r=.5]


%D The macro for typesetting the SlideTitle; this is adapted from a sample
%D document that Brooks Moses published on the wiki:

\define[1]\SlideTitle{\page\setlayer[slidetitle]%
     {\framed[frame=off,width=\textwidth,height=2.5cm,offset=0pt,top=\vss,bottom=\vss]{\switchtobodyfont[\TitleSize]\color[simpleslides:contrastcolor]{#1}}}}

%D The macro \tex{MakeTitle} produces a default title page with the author, the
%D title of the presentation, and the date. Using it is not mandatory.

\define\MakeTitle{%
\TitleBackground
\null
\vfill
\midaligned{\switchtobodyfont[\TitleSize]\color[simpleslides:contrastcolor]{\PRESTITtitle}}
\blank[2*line]
\midaligned{\color[simpleslides:contrastcolor]{\PRESTITauthor}}
\blank[3*line]
\midaligned{\color[simpleslides:contrastcolor]{\PRESTITdate}}
\vfill
\null}

%D The following parts are optional; if you don't use backgrounds and are
%D content with CONTEXT's default itemization, you don't have to set these
%D macros. 

%D We let Metapost calculate the background:

\startuseMPgraphic{left} 
StartPage ;
	fill Page withcolor \MPcolor{simpleslides:outercolor} ;
	path p ; pair pa ; pair pb ;
	p := Page enlarged (-15pt,-15pt) superellipsed .9 ;
	fill p withcolor \MPcolor{simpleslides:backgroundcolor} ;
	pickup pencircle scaled 20pt ;
	draw p withcolor \MPcolor{simpleslides:contrastcolor} ;
	if PageNumber>1:
		pa := point (3 + (6*PageNumber) / NOfPages) of p ;
		pb := point (3 + (6*(PageNumber-1)) / NOfPages) of p ; 
		draw (p cutafter pa) cutbefore pb withcolor \MPcolor{simpleslides:altcontrastcolor} ;
	fi ;
%	draw point 1 of p withcolor \MPcolor{simpleslides:altcontrastcolor} ;
%	draw point 8 of p withcolor blue ;
%	draw point 4 of p withcolor green ;
%	draw point 3 of p withcolor yellow ;
%	draw point 2 of p withcolor cyan ;
%	draw point 7 of p withcolor white ;
StopPage ;
\stopuseMPgraphic 

%D We define these backgrounds as overlays:

\defineoverlay 
[lecbackground] 
[\useMPgraphic{left}] 

%D These are shortcuts to switch backgrounds:

\define\HorizontalBackground{\setuplayout[header=1.5cm]\setupbackgrounds[page][background={lecbackground,slidetitle}]}
\define\TitleBackground{\setuplayout[header=0cm]\setupbackgrounds[page][background=lecbackground]}
\define\VerticalBackground{\setuplayout[header=0cm]\setupbackgrounds[page][background={lecbackground,slidetitle}]}
\define\NoBackground{\setupbackgrounds[page][background=nobackground]}

%D We use combinations for placing vertical pictures and text side by side, and
%D we want a distance of 1.1 cm between both.

\setupcombinations[distance=0cm]

%D The symbol for the first level of itemizations. 

\definesymbol[1][\useMPgraphic{ItSquare}]
\setupitemize[1][color=simpleslides:contrastcolor]

\protect
\stopmodule

\endinput

