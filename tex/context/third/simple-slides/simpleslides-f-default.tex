%D \module
%D   [      file=simpleslides-f-default,
%D        version=2007.07.15, 
%D          title=\CONTEXT\ Style File,
%D       subtitle=Presentation Module simpleslides --- Default font setup,
%D         author=Thomas A. Schmitz,
%D           date=\currentdate,
%D      copyright={Thomas A. Schmitz}]
%C
%C Copyright 2007 Thomas A. Schmitz.
%C This file may be distributed under the GNU General Public License v. 2.0.

\writestatus{simpleslides}{loading default font setup}

\startmodule[simpleslides-f-default]

\unprotect

\setupbodyfontenvironment[default][em=italic] 

%D The fontsize is set via the \type{size}||key; it will be used in numerous
%D setup||commands. In earlier versions, I had used the \tex{processaction}
%D mechanism to define the \tex{NormalSize} and \tex{TitleSize}, but Aditya
%D rightly pointed out that this is somewhat inflexible. I now set the font
%D dimensions directly; \tex{TitleSize} is calculated from \tex{NormalSize}. I
%D do a few tests to get nice sizes. 

\newdimen\simpleslidesNormalSize  
\newdimen\simpleslidesTitleSize  

\simpleslidesNormalSize=\moduleparameter{simpleslides}{size}\relax

\ifdim\simpleslidesNormalSize<16pt%
	\simpleslidesTitleSize=1.6\simpleslidesNormalSize\relax%
\else%
	\ifdim\simpleslidesNormalSize<20pt%
		\simpleslidesTitleSize=1.4142\simpleslidesNormalSize\relax%
	\else%
		\simpleslidesTitleSize=30pt\relax%
	\fi%
\fi%

\def\NormalSize{\the\simpleslidesNormalSize}
\def\TitleSize {\the\simpleslidesTitleSize}

\beginOLDTEX 
\setupencoding[default=ec]
\endOLDTEX

%D The bodyfont needs to be defined so \CONTEXT\ can calculate size switches,
%D math formulas, etc.

\starttypescript [serif] [default] [size] 
\definebodyfont [14pt,15pt,16pt,17pt,20pt,25pt,\NormalSize,\TitleSize] [rm] [default] 
\stoptypescript

\starttypescript [sans] [default] [size] 
\definebodyfont [14pt,15pt,16pt,17pt,20pt,25pt,\NormalSize,\TitleSize] [ss] [default] 
\stoptypescript

\definebodyfontenvironment[\NormalSize]
\definebodyfontenvironment[\TitleSize]

\protect

\stopmodule
