%D \module
%D   [      file=simpleslides-s-default,
%D        version=2007.07.15, 
%D          title=\CONTEXT\ Style File,
%D       subtitle=Presentation Module simpleslides --- Default style setup,
%D         author=Thomas A. Schmitz \& Aditya Mahajan,
%D           date=\currentdate,
%D      copyright={Thomas A. Schmitz}]
%C
%C Copyright 2007 Thomas A. Schmitz.
%C This file may be distributed under the GNU General Public License v. 2.0.

\writestatus{simpleslides}{loading default style setup}

\startmodule[simpleslides-s-default]

\unprotect

%D This module is the first sub-module that is loaded by
%D \filename{simpleslides}.  This sets up the style macros for the module. We
%D choose a rather plain style as the default; separate style sub-modules
%D redefine some internal macros to achieve fancier effects.


%D We start with the page layout. S6 has the ratio of a usual computer screen.

\setuppapersize[S6][S6]

%D We do not want page numbers, but we sometimes want to use the page numbering
%D mechanism, so we make sure pages are counted, but the numbers are not
%D displayed. 

\setuppagenumbering[location=]

%D As for the page layout: most of the numbers have been reached by
%D trial and error; I have just taken what seemed to produce the best output.

\setuplayout [width=fit,
              margin=0.6cm,
              height=fit, 
              header=0.15cm, 
              footer=1.35cm, 
	      footerdistance=0.5cm,
              topspace=0.5cm, 
              backspace=1cm,
              location=singlesided]


%D We define some other layouts which can be used to change the layout of
%D specific kinds of pages. This module defines three kinds of pages:
%D horizontal, vertical, and title, and we allow the possibility of changing the
%D layout of all three page styles.

\definelayout [simpleslides:layout:horizontal]
\definelayout [simpleslides:layout:vertical]
\definelayout [simpleslides:layout:title]

%D We want colored presentations

\setupcolors[state=start]

%D And colored typesetting for \TEX\ code.

\definetype[typeTEX][option=color]

%D Presentations have relatively dense lines, we do not care about underfull
%D lines, but allow emergency stretch.

\setuptolerance[verytolerant,stretch] 

%D We use combinations for placing vertical pictures and text side by side, and
%D we want a distance of 1.1 cm between both.

\setupcombinations[distance=1.1cm]

%D Next we define some dimensions that are used as nominal values by other
%D macros and the user.

\define\NormalHeight        {.830\textheight}
\define\NormalWidth         {.476\textwidth}
\define\PictureFrameHeight  {.830\textheight}
\define\PictureFrameWidth   {.476\textwidth}

%D As we said before, the presentation consists of three kinds of pages: title,
%D horizontal, and vertical. Each page can have a different background; we can
%D switch to a specific background using
%D \starttyping
%D \setupPageBackground[horizontal]
%D \stoptyping

\def\setupPageBackground%
   {\dosingleargument\dosetupPageBackground}

\def\dosetupPageBackground[#1]%
  {\setups{simpleslides:background:#1}}

%D The backgrounds in all the styles are very similar. So, we define overlays to
%D capture the different features of the backgrounds, and then simply define the
%D overlays in different styles.

\defineoverlay[simpleslides:background:horizontal]
\defineoverlay[simpleslides:background:title]
\defineoverlay[simpleslides:background:vertical]
\defineoverlay[simpleslides:background:ornament]

\defineoverlay[simpleslides:background:highlight]
              [\useMPgraphic{simpleslides:MP:highlight}]%

\definelayer[simpleslides:layer:slidetitle]
    [\c!width=\paperwidth,
    \c!height=\paperheight]

%D A generic frame which is used for titles and other commands

\defineframed[simpleslides:framed]
             [frame=off,offset=0pt,
              top=\vss,bottom=\vss]

%D Now we define setups for specific backgrounds

\startsetups simpleslides:background:title
   \setuplayout[simpleslides:layout:title]
     \setupbackgrounds[\v!page]
        [background={simpleslides:background:title}]
\stopsetups

\startsetups simpleslides:background:horizontal
  \setuplayout[simpleslides:layout:horizontal]
  \setupbackgrounds[\v!page]
        [background={simpleslides:background:horizontal,
                     simpleslides:background:ornament,
                     simpleslides:layer:slidetitle}]
\stopsetups

\startsetups simpleslides:background:vertical
  \setuplayout[simpleslides:layout:vertical]
  \setupbackgrounds[\v!page]
        [background={simpleslides:background:vertical,
                     simpleslides:background:ornament}]
\stopsetups

%D We also define some fall backs for presentation backgrounds.

\startsetups simpleslides:background:default
  \setups{simpleslides:background:horizontal}
\stopsetups

\startsetups simpleslides:background:none
  \setupbackgrounds[\v!page]
                   [background=]
\stopsetups

\startsetups simpleslides:background:empty
   \message{I don't know the name of the background you have provided.}
   \setups{simpleslides:background:horizontal}
\stopsetups
  
%D Now we move on to specific components of the presentation. The first thing in
%D any presentation is the title page, so we start with that. We define a macro
%D that stores values for the title, author and date of the presentation. These
%D are input as
%D \starttyping
%D \setupTitle[title={How to write a sub module for simpleslides},
%D             author={Aditya Mahajan},
%D             date={\currentdate[d=10,m=7,y=2008]}]
%D \stoptyping
%D This setup command can also setup the color and style for the title, author,
%D and date.

\def\setupTitle%
  {\dosingleargument\dosetupTitle}

\def\dosetupTitle[#1]%
  {\setupmodule[simpleslides:title][#1]}

%D A few macros to save some typing

\def\simpleslidestitleparameter%#1
  {\moduleparameter{simpleslides:title}}

\def\simpleslidestitlecomponent#1%
  {\simpleslidestitleparameter{\c!before#1}
      \startalignment[\simpleslidestitleparameter{#1\c!align}]
        \doattributes{\??md:simpleslides:title:}{#1\c!style}{#1\c!color}
          {\moduleparameter{simpleslides:title}{#1}}
       \stopalignment
    \simpleslidestitleparameter{\c!after#1}}

%D \macros{placeTitle}
%D
%D The macro \tex{placeTitle} produces a title page with the author, the
%D title of the presentation, and the date. Using it is not mandatory.

\define\placeTitle
  {\page
   \setupPageBackground[title]
   \null
   \simpleslidestitleparameter\c!before
   \startalignment[\simpleslidestitleparameter\c!align]
   \dostartattributes{\??md:simpleslides:title:}\c!headstyle\c!headcolor
    \simpleslidestitlecomponent\c!title
    \simpleslidestitlecomponent\c!author
    \simpleslidestitlecomponent\c!date
   \dostopattributes
   \stopalignment
   \simpleslidestitleparameter\c!after
   \page}

%D If a sub-module wants to display more information about the presentation
%D (e.g., institute of the author), it can be simply input as
%D \starttyping
%D   \setupTitle[institute={Some University}]
%D \stoptyping
%D This parameter can be accessed inside the module as
%D \type{\simpleslidestitleparameter{institute}}, and the module writer
%D can redefine \type{\placeTitle} to display the institute at the appropriate
%D place. 


%D Now we setup some default values for the title page. Other modules will
%D redefine these values later.

\setupTitle
  [\c!title=,
   \c!author=,
   \c!date=\currentdate,
   \c!style=,
   \c!color=red,
   \c!align=\v!middle,
   \c!before=\vfill,
   \c!after=\vfill,
   \c!title\c!style={\switchtobodyfont[\TitleSize]},
   \c!title\c!color=,
   \c!title\c!align=,%\v!middle,
   \c!author\c!style=,
   \c!author\c!color=,
   \c!author\c!align=,%\v!middle, 
   \c!date\c!style=,
   \c!date\c!color=,
   \c!date\c!align=,%\v!middle,
   \c!before\c!title=,
   \c!before\c!author=,
   \c!before\c!date=,
   \c!after\c!title={\blank[2*line]},
   \c!after\c!author={\blank[3*line]},
   \c!after\c!date=]

%D \macros{SlideTitle}
%D \tex{SlideTitle}: well, the name says it all. The
%D argument is typeset as the title, but the implementation and the result
%D (alignment, size, distance to text, color etc.) vary from module to module,
%D so this macro is defined in each of the submodules. 

\def\setupSlideTitle
  {\dosingleargument\dosetupSlideTitle}

\def\dosetupSlideTitle[#1]%
  {\setupmodule[simpleslides:slidetitle]
               [\c!alternative=\v!normal,\c!width=\textwidth,
                \c!height=2\lineheight,#1]}

\def\simpleslidesslidetitleparameter%#1
  {\moduleparameter{simpleslides:slidetitle}}

\def\defineSlideTitleAlternative
  {\dodoubleargument\dodefineSlideTitleAlternative}

\def\dodefineSlideTitleAlternative[#1][#2]% #3
  {\setvalue{simpleslides:slidetitle:#1}}

\defineSlideTitleAlternative[\v!normal]#1%
  {\page[\simpleslidesslidetitleparameter\c!page]%
   \simpleslidesslidetitleparameter\c!before
      \startalignment[\simpleslidesslidetitleparameter\c!align]
        \doattributes{\??md:simpleslides:slidetitle:}\c!style\c!color{#1}%
       \stopalignment
    \simpleslidesslidetitleparameter\c!after}

%D The layered alternative is adapted from a sample document that Brooks Moses
%D published on the wiki:

\defineSlideTitleAlternative[layer]#1%
  {\page[\simpleslidesslidetitleparameter\c!page]%
   \simpleslidesslidetitleparameter\c!before
   \setlayer[simpleslides:layer:slidetitle]%
     {\getvalue{simpleslides:framed}
        [\c!width=\simpleslidesslidetitleparameter\c!width,
        \c!height=\simpleslidesslidetitleparameter\c!height,
         \c!align=\simpleslidesslidetitleparameter\c!align]
        {\doattributes{\??md:simpleslides:slidetitle:}\c!style\c!color{#1}}}%
    \simpleslidesslidetitleparameter\c!after}


%D Now the main user command

\def\SlideTitle{\dowithpargument\doSlideTitle}

\def\doSlideTitle% #1
  {\getvalue{simpleslides:slidetitle:\simpleslidesslidetitleparameter{alternative}}}

\setupSlideTitle
  [\c!page=\v!yes,
   \c!style={\switchtobodyfont[\TitleSize]},
   \c!before={\setupPageBackground[horizontal]},
   \c!after={\blank[0.75cm]},
   \c!align=\v!middle,
   \c!alternative=\v!normal]


%AM: Clean this up later.

%D \macros{IncludePicture}
%D The macros for placing pictures. This section has been entirely rewritten in
%D order to provide a cleaner user interface and to make the code easier to
%D maintain. We now have just one macro, \tex{IncludePicture}, which
%D automatically inserts page breaks and adjusts the background and margins, if
%D necessary. This macro takes four arguments: 
%D \setup{IncludePicture} 
%D The first argument decides whether the picture will be place in a horizontal
%D or vertical arrangement; the second argument is the filename of the picture
%D you want to include; the third argument does some setup, and the fourth
%D argument (in braces) is the text accompanying the picture, which will be
%D placed either in a \tex{SlideTitle} environment (for horizontal pictures) or
%D opposite the picture, centered horizontally and vertically, for vertical
%D pictures.

%D The user macro takes three arguments in brackets + an additional argument
%D for typeset content. So we define it in two steps:

\def\setupPicture
  {\dosingleargument\dosetupPicture}

\def\dosetupPicture[#1]%
  {\setupmodule[simpleslides:picture]
               [highlight=\v!no,
                steps=5,
                opacity=0.5,
                \c!x\c!scale=couple,
                \c!y\c!scale=couple,
                \c!width=,\c!height=,#1]}

\setupPicture
  [highlight=\v!no,%yes|no|default=no
   \c!alternative=\v!none,%none|circle|arrow|focus|default=none
   \c!grid\c!color=darkgreen,
   \c!grid=\v!no,
   \c!sub\c!grid=\v!no,
   steps=5,%controls the number of subdivisions of the subgrid
   \c!rulethickness={OverlayWidth/100},%for circle and arrow
   \c!color=orange,% should this be called highlightcolor?
   \c!x=5,
   \c!y=5,
   \c!x\c!scale=couple,
   \c!y\c!scale=couple,
   length=2cm,
   direction=0,
   opacity=0.5,
   shadow=\v!no,
   shadow\c!color=black,
  ]

\def\simpleslidesPictureParameter%#1
  {\moduleparameter{simpleslides:picture}}


\newconditional\simpleslidesHighlight   \setfalse\simpleslidesHighlight
\newconditional\simpleslidesGrid        \setfalse\simpleslidesGrid
\newconditional\simpleslidesSubGrid     \setfalse\simpleslidesSubGrid

\chardef\simpleslidesHighlightAlternative\zerocount %0=none, 1=circle, 
                                                    %2=arrow, 3=focus
\chardef\simpleslidesShadow\zerocount %0=no, 1=topleft, 2=topright,
                                      %3=bottomright, 4=bottomleft

\def\IncludePicture%
   {\dotripleargument\doIncludePicture}


\def\doIncludePicture[#1][#2][#3]#4% dir graphic options text
  {\begingroup % Else the settings of \setupPicture are global 
   \setupPicture[#3]%
   \processaction
     [\simpleslidesPictureParameter{highlight}]
     [    \v!yes=>\settrue\simpleslidesHighlight,
           \v!no=>\setfalse\simpleslidesHighlight,
      \v!default=>\setfalse\simpleslidesHighlight,
      \v!unknown=>\setfalse\simpleslidesHighlight]
   %
   \processaction
     [\simpleslidesPictureParameter\c!grid]
     [    \v!yes=>\settrue\simpleslidesGrid,
           \v!no=>\setfalse\simpleslidesGrid,
      \v!default=>\setfalse\simpleslidesGrid,
      \v!unknown=>\setfalse\simpleslidesGrid]
   %
   \processaction
     [\simpleslidesPictureParameter{\c!sub\c!grid}]
     [    \v!yes=>\settrue\simpleslidesSubGrid,
           \v!no=>\setfalse\simpleslidesSubGrid,
      \v!default=>\setfalse\simpleslidesSubGrid,
      \v!unknown=>\setfalse\simpleslidesSubGrid]
   %
   \processaction
     [\simpleslidesPictureParameter{shadow}]
       [     \v!no=>\chardef\simpleslidesShadow\zerocount,
            \v!yes=>\chardef\simpleslidesShadow\plusthree,
     \v!top\v!left=>\chardef\simpleslidesShadow\plusone,
    \v!top\v!right=>\chardef\simpleslidesShadow\plustwo,
 \v!bottom\v!right=>\chardef\simpleslidesShadow\plusthree,
  \v!bottom\v!left=>\chardef\simpleslidesShadow\plusfour,
        \v!default=>\chardef\simpleslidesShadow\zerocount,
        \v!unknown=>\chardef\simpleslidesShadow\zerocount]
   %
   \edef\simpleslidesGridColor{\simpleslidesPictureParameter{\c!grid\c!color}}%
   \doifnothing{\simpleslidesGridColor}{\edef\simpleslidesGridColor{green}}%
   %
   \processaction
     [\simpleslidesPictureParameter\c!alternative]
     [   circle=>\chardef\simpleslidesHighlightAlternative\plusone
                  \simpleslidesSetEllipseDimensions,
           arrow=>\chardef\simpleslidesHighlightAlternative\plustwo,
           focus=>\chardef\simpleslidesHighlightAlternative\plusthree
                  \simpleslidesSetEllipseDimensions,
      \v!default=>\chardef\simpleslidesHighlightAlternative\zerocount,
      \v!unknown=>\chardef\simpleslidesHighlightAlternative\zerocount]
   %
    \processaction[#1]
      [\v!horizontal=>{\doIncludePictureHorizontal[#2]{#4}},
         \v!vertical=>{\doIncludePictureVertical[#2]{#4}},
          \v!default=>{\doIncludePictureHorizontal[#2]{#4}},
          \v!unknown=>{\doIncludePictureHorizontal[#2]{#4}}%
                       \message{Keywords for positioning pictures are
                         "horizontal" or "vertical." 
                          Please specify one of them!}]
     \endgroup}

%D We have to set the width or height for vertical and horizontal pictures
%D separately. We test if the user has supplied values for the "width" and
%D "height" key. If they have been set, these are used for the width/height of
%D the picture. If not, we determine the ratio of the picture and let it fill
%D the available space. Small pictures will be enlarged, big pictures will be
%D fit. This code was suggested by Wolfgang Schuster on the mailing list.

\def\setPictureDimensionsTo#1#2% graphic width 
  {\doifelsenothing{\simpleslidesPictureParameter\c!width}
     {\doifelsenothing{\simpleslidesPictureParameter\c!height}
        {\setbox\scratchbox\hbox{\externalfigure[#1]}
        % Courtsey Wolfgang Schuster
        \ifdim\dimexpr#2*100/\wd\scratchbox\relax < 
              \dimexpr\PictureFrameHeight*100/\ht\scratchbox\relax
          %
          \setvalue{simpleslides:picture:width}{#2}%
          \setvalue{simpleslides:picture:height}{}%
        \else
          \setvalue{simpleslides:picture:width}{}%
          \setvalue{simpleslides:picture:height}{\PictureFrameHeight}%
        \fi}
     {\setvalue{simpleslides:picture:width}{}%
      \setvalue{simpleslides:picture:height}{\simpleslidesPictureParameter\c!height}}}
 {\setvalue{simpleslides:picture:width}{\simpleslidesPictureParameter\c!width}
  \setvalue{simpleslides:picture:height}{}}}% not \simpleslidesPictureParameter\c!height

%D This macro places the picture centered both vertically and horizontally. 

\def\doplacePictureFrame[#1]#2%
  {\framed[\c!frame=\v!off,
           \c!framecolor=green,
           \c!align=\v!middle,
           \c!top=\vss,
           \c!bottom=\vss,
           \c!strut=\v!no,
           \c!offset=\zeropoint,#1]
      {\framed[\c!frame=\v!off,
		       \c!framecolor=red,
               \c!strut=\v!no,
               \c!offset=\zeropoint,
               \c!width=\v!fit,
               \c!height=\v!fit,
               \c!background={foreground,simpleslides:background:highlight}]
      {#2}}}


%D This macro places the picture in horizontal mode.

\def\doIncludePictureHorizontal[#1]#2% graphic text
 {\setPictureDimensionsTo{#1}\textwidth
  \SlideTitle{#2}%
  \doplacePictureFrame
    [\c!width=\textwidth,
     \c!height=\PictureFrameHeight]
    {\externalfigure[#1]
                    [\c!height=\getvalue{simpleslides:picture:height},
                      \c!width=\getvalue{simpleslides:picture:width}]}%
    \page}% Otherwise last page is not correct in \MKII.

%D This macro places the picture in vertical mode.

\def\doIncludePictureVertical[#1]#2% graphic text
  {\page
   \setPictureDimensionsTo{#1}\PictureFrameWidth
    \setupPageBackground[vertical]
    \startcombination[2]
      {\doplacePictureFrame
        [\c!height=\textheight,
         \c!width=\PictureFrameWidth]
         {\externalfigure[#1]
                    [\c!height=\getvalue{simpleslides:picture:height},
                      \c!width=\getvalue{simpleslides:picture:width}]}}{}
       {\framed[\c!frame=\v!off,
                \c!height=\textheight,
                \c!width=\PictureFrameWidth,
                \c!top=\vss,
                \c!bottom=\vss,
                \c!align=\v!middle,
                \c!strut=\v!no]{#2}}{}
    \stopcombination
    \page}% Otherwise last page is not correct in \MKII.
   
%D Now, we define the main \METAPOST\ graphic to highlight the picture. It first
%D checks if the user has asked for a grid, and if so, draws a grid and a
%D subgrid. Then it checks which alternative (arrow, circle, focus) is requested
%D and draws that. Notice that we have intermixed \TEX\ and \METAPOST, 
%D exploiting the power of the MP-TEX interaction provided by \CONTEXT.

\startuseMPgraphic{simpleslides:MP:highlight}
\ifconditional \simpleslidesHighlight
  \ifconditional \simpleslidesGrid
	\includeMPgraphic{simpleslides:MP:highlight:grid}
    \ifconditional \simpleslidesSubGrid
      \includeMPgraphic{simpleslides:MP:highlight:subgrid}%
    \fi
  \fi
 \ifcase \simpleslidesHighlightAlternative \or
  \includeMPgraphic{simpleslides:MP:highligh:circle} \or
  \includeMPgraphic{simpleslides:MP:highligh:arrow}  \or
  \includeMPgraphic{simpleslides:MP:highligh:focus}  \fi
  \fi
  setbounds currentpicture to OverlayBox ;
\stopuseMPgraphic

%D Metapost graphic to draw a 10x10 grid on a picture. The width of the grid
%D line is 1/250th of the width of the picture, so smaller picture have smaller
%D grid lines.

\startuseMPgraphic{simpleslides:MP:highlight:grid}
  save xaxis, yaxis ; path xaxis, yaxis ;
  xaxis := leftboundary OverlayBox ;
  yaxis := bottomboundary OverlayBox ;


  for i = 0 upto 10 :

    draw xaxis shifted (i/10*OverlayWidth, 0) 
               withpen pencircle scaled (OverlayWidth/250) 
               withcolor \MPcolor\simpleslidesGridColor ;

    draw yaxis shifted (0,i/10*OverlayHeight) 
               withpen pencircle scaled (OverlayWidth/250) 
               withcolor \MPcolor\simpleslidesGridColor ;

    label.bot(textext("\switchtobodyfont[9pt]\startcolor[simpleslides:textcolor]" 
                     & decimal i & "\stopcolor"),
              (i/10*OverlayWidth, 0)) ;
    label.top(textext("\switchtobodyfont[9pt]\startcolor[simpleslides:textcolor]" 
                      & decimal i & "\stopcolor"),
              (i/10*OverlayWidth, OverlayHeight)) ;

    label.lft(textext("\switchtobodyfont[9pt]\startcolor[simpleslides:textcolor]" 
                     & decimal i & "\stopcolor"),
              (0, i/10*OverlayHeight)) ;
    label.rt(textext("\switchtobodyfont[9pt]\startcolor[simpleslides:textcolor]" 
                    & decimal i & "\stopcolor"),
              (OverlayWidth, i/10*OverlayHeight)) ;
  endfor ;
  setbounds currentpicture to OverlayBox ;
\stopuseMPgraphic

%D The Metapost graphic draws a fine grid on the picture. The number of grid
%D lines is controlled by \type{steps} parameter of \type{\setupPicture}. Each
%D grid box of the 10x10 grid drawn above is subdivided into \type{steps} number
%D of finer lines. 

\startuseMPgraphic{simpleslides:MP:highlight:subgrid}
  save xaxis, yaxis ; path xaxis, yaxis ;
  xaxis := leftboundary OverlayBox ;
  yaxis := bottomboundary OverlayBox ;


  save stepSize ; numeric stepSize ;
  stepSize := \simpleslidesPictureParameter{steps} + epsilon ;

  for i = 0 step 10/stepSize until 100 :
    if i mod 10 <> 0 :
      draw xaxis shifted (i/100*OverlayWidth, 0) 
                 withpen pencircle scaled (OverlayWidth/500) 
                 withcolor \MPcolor\simpleslidesGridColor ;
      draw yaxis shifted (0,i/100*OverlayHeight) 
                 withpen pencircle scaled (OverlayWidth/500) 
                 withcolor \MPcolor\simpleslidesGridColor ;
    fi 

  endfor ;
  setbounds currentpicture to OverlayBox ;
\stopuseMPgraphic


%D This is a helper macro to get the dimensions for the circle. Taking care of
%D expansion inside Metapost is tricky, so we define an external macro to set
%D the x and y scales correctly. This macro does not check if both x and y are
%D equal to 'couple'. We should probably catch that error inside Metapost, and
%D give an error message on the terminal if that is the case.


\def\simpleslidesSetEllipseDimensions%
  {\edef\simpleslidesCurrentXscale{\simpleslidesPictureParameter{\c!x\c!scale}}
   \edef\simpleslidesCurrentYscale{\simpleslidesPictureParameter{\c!y\c!scale}}
   \doifelse{\simpleslidesCurrentXscale}{couple}
     {\doifelse{\simpleslidesCurrentYscale}{couple}
       {\edef\simpleslidesCurrentXscale{1}
        \edef\simpleslidesCurrentYscale{1}}
       {\dosimpleslidesSetEllipseDimensions}}
    {\dosimpleslidesSetEllipseDimensions}}

\def\dosimpleslidesSetEllipseDimensions%
  {\processaction
     [\simpleslidesCurrentXscale]
     [    couple=>\edef\simpleslidesCurrentXscale{\simpleslidesCurrentYscale*OverlayHeight/5},
      \v!unknown=>\edef\simpleslidesCurrentXscale{\simpleslidesCurrentXscale*OverlayWidth/5}]
   \processaction
     [\simpleslidesCurrentYscale]
     [    couple=>\edef\simpleslidesCurrentYscale{\simpleslidesCurrentXscale}, % no *OverlayWidth/5, it has already been scaled
      \v!unknown=>\edef\simpleslidesCurrentYscale{\simpleslidesCurrentYscale*OverlayHeight/5}]}

%D Metapost graphic to draw a circle. The circle is centered at (x,y) with a
%D x-radius equal to \type{xscale} and y-radius equal to \type{yscale}. If
%D either \type{xscale} or \type{yscale} equals \type{couple}, a circle is drawn
%D with radius equal to the other (non-couple) scale parameter. The thickness of
%D the circle is \type{rulethickness} and the color is \type{color}. The
%D circle is rotated by an angle \type{direction}

\startuseMPgraphic{simpleslides:MP:highligh:circle}
  save p; path p;
  p := fullcircle xyscaled 
        % The dimensions are divided by 5 rather than 10, since the scale 
        % measures the radius, rather than the diameter.
        (\simpleslidesCurrentXscale, \simpleslidesCurrentYscale)
       rotated (\simpleslidesPictureParameter\c!direction) 
       shifted
       (\simpleslidesPictureParameter\c!x *OverlayWidth/10,
        \simpleslidesPictureParameter\c!y *OverlayHeight/10)  ;

  \ifcase \simpleslidesShadow \else
  
  save shadow_direction ; pair shadow_direction ;
  shadow_direction =\simpleslidesPictureParameter\c!rulethickness * 
     \ifcase \simpleslidesShadow \or
          (-1,1) \or (1,1) \or (1,-1) \or (-1,-1) \fi ;


  save shadow_thickness ; numeric shadow_thickness ;
  shadow_thickness := 2.7\simpleslidesPictureParameter\c!rulethickness ;

  def shadow_draw = draw enddef ;

  \includeMPgraphic{simpleslides:MP:highligh:shadow}
  \fi

  draw p withcolor \MPcolor{\simpleslidesPictureParameter\c!color} 
       withpen pencircle scaled (\simpleslidesPictureParameter\c!rulethickness);

\stopuseMPgraphic

%D Metapost graphic to draw an arrow. The arrow's tip is at \type{(x,y)}, and it
%D is \type{length} long in the direction specified by \type{direction}. The
%D arrow is \type{2×rulethickness} thick (the factor of 2 is there so that the
%D same default works for both circle and arrows) and is on \type{color} color.
%D
%D TODO: Add \type{shadow} key.

\startuseMPgraphic{simpleslides:MP:highligh:arrow}
  autoarrows := true ; % so that arrow heads are scaled correctly.

  z1 = (\simpleslidesPictureParameter\c!x *OverlayWidth/10,
        \simpleslidesPictureParameter\c!y *OverlayHeight/10)   ;
  z2 = z1 + \simpleslidesPictureParameter{length} *
            dir(\simpleslidesPictureParameter{direction}) ;

  save p ; path p ;
  p := z2 -- z1 ;

  \ifcase \simpleslidesShadow \else
  save shadow_direction ; pair shadow_direction ;
  shadow_direction =2*\simpleslidesPictureParameter\c!rulethickness * 
     dir(\simpleslidesPictureParameter{direction} + 
     \ifcase \simpleslidesShadow \or 
          135 \or 45 \or -45 \or -135 \fi ) ;

  save shadow_thickness ; numeric shadow_thickness ;
  shadow_thickness := 2*\simpleslidesPictureParameter\c!rulethickness ;

  def shadow_draw = drawarrow enddef ;

  \includeMPgraphic{simpleslides:MP:highligh:shadow}
  \fi

  drawarrow p
       withcolor \MPcolor{\simpleslidesPictureParameter\c!color} 
       withpen pencircle scaled (2*\simpleslidesPictureParameter\c!rulethickness);

\stopuseMPgraphic

%D Metapost graphic to focus on a particular area. The coordinates of the focus
%D area are equivalent to that of \type{alternative=circle}. The "unfocussed"
%D area is covered with \type{shadowcolor} made transparent by
%D \type{opacity} amount.

\startuseMPgraphic{simpleslides:MP:highligh:focus}
  save p; path p;
  p := fullcircle xyscaled 
        % The dimensions are divided by 5 rather than 10, since the scale 
        % measures the radius, rather than the diameter.
        (\simpleslidesCurrentXscale, \simpleslidesCurrentYscale)
       rotated (\simpleslidesPictureParameter\c!direction) 
       shifted
       (\simpleslidesPictureParameter\c!x *OverlayWidth/10,
        \simpleslidesPictureParameter\c!y *OverlayHeight/10)  ;

  fill OverlayBox -- reverse p -- cycle
       withcolor transparent ("normal",
         \simpleslidesPictureParameter{opacity},
         \MPcolor{\simpleslidesPictureParameter{shadow\c!color}}) ;
\stopuseMPgraphic

%D The next Metapost graphic draws an arrow. To ensure code re-usability, we
%D assume that the shape of the shadow is stored as a path \type{p}, it is
%D shifted by amount \type{shadow_direction} and its thickness is
%D \type{shadow_thickness}. It is the responsibility of the calling function to
%D ensure that this is done. There are four options for the shadow,
%D \type{topleft}, \type{topright}, \type{bottomright}, and \type{bottomleft}.
%D (TODO: Keep this thickness configurable). The shadow is drawn in color
%D \type{shadowcolor}. The arrow is drawn with a transparent color whose opacity
%D is given by \type{opacity}.

\startuseMPgraphic{simpleslides:MP:highligh:shadow}
  %% I hope there were a better way to do this.
  for i = 1 upto 20 :
  shadow_draw p shifted shadow_direction 
       withcolor transparent ("normal",
         \simpleslidesPictureParameter{opacity}/10, % 0.1,
         \MPcolor{\simpleslidesPictureParameter{shadow\c!color}}) 
       withpen pencircle scaled (i*shadow_thickness/20) ;
  endfor 
\stopuseMPgraphic


%D This is a small square which will be used for itemizations; it will be
%D placed in the margin. 

\startuniqueMPgraphic{simpleslides:itemize:triangle}
  fill (0,0) -- (0,0.4cm) -- (0.6cm,0.2cm) -- cycle 
       withcolor \MPcolor{simpleslides:itemize:color} ;
\stopuniqueMPgraphic

\startuniqueMPgraphic{simpleslides:itemize:square}
  fill unitsquare xyscaled(0.4cm,0.4cm) 
       withcolor \MPcolor{simpleslides:itemize:color} ;
\stopuniqueMPgraphic


\protect

\stopmodule
