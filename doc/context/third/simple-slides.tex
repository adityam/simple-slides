\setupcolors[state=start]

\setuphead[title][alternative=middle] % Will make it fancy later

\starttext

\title
  Simple Slides \\
  A \CONTEXT\ module for making slides \\

\section Introduction

This module aims to provide an easy|-|to|-|use, simple, and consistent interface
for writing simple slides|/|presentations in \CONTEXT. Thomas had the idea to
create such a module while preparing his course presentations with \CONTEXT. He
wanted to change the visual style of his presentations without changing the
source files, so he wrote different styles that followed the same logic and
provided the same macros. These styles later evolved into the
\filename{taspresent} module. Thomas presented the internals of the
\filename{taspresent} module in the second \CONTEXT\ user meeting at Bohinj, and
in the ensuing discussions, Aditya and Thomas decided to modularize and
\quotation{\CONTEXT{ize}} some of the internals of the module, giving rise to
the current module. 

The salient feature of this module are:
\startitemize[intro]
  \item The module is meant for the presentations which will be shown on a
    digital projector. They have no interactive elements (such as buttons or
    hyperlinks) and no navigational tools (such as table of contents).
  \item The module comes with several predefined styles. These styles are sober
    in appearance meant for an academic presentation. It also provides some
    macros to help in presenting slides with both pictures and text.
  \item Most styles allow for some degree of user|-|reconfigurability. Designing
    a new style is also easy.
\stopitemize

This module provides a simple stricture that will be suitable for beginning or
intermediate users of \CONTEXT, or someone who does not want to spend too much
time playing around with different configuration options for \CONTEXT. As such
it focusses on different users than Han's presentation modules which have larger
and fancier features. This module also offers much less features than the
\LATEX\ \filename{beamer} package. Its main strength is its ease of use; you
should be able to write your first presentation after spending five minutes with
this manual.

\stoptext
