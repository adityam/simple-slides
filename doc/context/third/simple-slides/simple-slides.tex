\setupcolors     [state=start]
\setupinteraction[state=start,style=normal]

%% Layout : <<<
\setuplayout[
                   width=middle,
                  height=middle,
                %location=middle,
                topspace=0.5in,
             bottomspace=.75in,
          bottomdistance=.25in,
                  bottom=.25in,
               backspace=1.0in,
                cutspace=1.0in,
              leftmargin=0.55in,
             rightmargin=0.55in,
      leftmargindistance=0.1in,
     rightmargindistance=0.1in,
                  header=0.25in,
                  footer=0.5in,
           headerdistace=0.25in,
          footerdistance=0.25in,
                 marking=on,
%                   grid=yes,
    ]

\setuppagenumbering [location=footer]

%% >>>
%% Typescripts : <<<
\starttypescript [serif] [dejavu]
    \setups[font:fallback:sans]
    \definefontsynonym [Serif]           [name:DejaVu Serif] 
                       [features=default]
    \definefontsynonym [SerifItalic]     [name:DejaVu Serif Italic]    
                       [features=default]
    \definefontsynonym [SerifBold]       [name:DejaVu Serif Bold]       [features=default]
    \definefontsynonym [SerifBoldItalic] [name:DejaVu Serif Bold Italic]
                       [features=default]
\stoptypescript

\starttypescript [sans] [dejavu]
    \setups[font:fallback:sans]
    \definefontsynonym [Sans]           [name:DejaVu Sans]    
                       [features=default]
    \definefontsynonym [SansItalic]     [name:DejaVu Sans Oblique]    
                       [features=default]
    \definefontsynonym [SansBold]       [name:DejaVu Sans Bold]
                       [features=default]
    \definefontsynonym [SansBoldItalic] [name:DejaVu Sans Bold Oblique] 
                       [features=default]
\stoptypescript

\starttypescript [mono] [dejavu]
    \setups[font:fallback:mono]
    \definefontsynonym [Mono]           [name:DejaVu Sans Mono]   
                       [features=default]
    \definefontsynonym [MonoItalic]     [name:DejaVu Sans Mono Oblique]    
                       [features=default]
    \definefontsynonym [MonoBold]       [name:DejaVu Sans Mono Bold]      
                       [features=default]
    \definefontsynonym [MonoBoldItalic] [name:DejaVu Sans Mono Bold Oblique]
                       [features=default]
\stoptypescript

\starttypescript [dejavu]
    \definetypeface [dejavu] [rm] [serif] [dejavu] [default]
    \definetypeface [dejavu] [ss] [sans]  [dejavu] [default]
    \definetypeface [dejavu] [tt] [mono]  [dejavu] [default]
    \definetypeface [dejavu] [mm] [math]  [modern] [default]
\stoptypescript

\usetypescript[dejavu]
\setupbodyfont[dejavu,10pt]

%% >>>
%% Logos: <<<
\logo [TEX]      {Tex}
\logo [LATEX]    {Latex}
\logo [CONTEXT]  {Context}
\logo [PDFTEX]   {pdftex}
\logo [LUATEX]   {Luatex}
\logo [XETEX]    {Xetex}
\logo [MKII]     {MkII}
\logo [MKIV]     {MkIV}

\setupsorting[logo][style=normal]

%% >>>

\definetype[typeTEX][option=tex, style=type]
\definetype[command][color=darkred, style=type]
\definetype[options][color=darkblue, style=type]

\setupindenting[medium,yes]
\setupwhitespace[medium]

\setuphead[title][alternative=middle, textstyle=sansbold] 
\setuphead[section,subsubject,subsection]
          [numberstyle=sansbold,textstyle=sansbold]

\setuplistalternative[a]
    [distance=0pt,width=1em,stretch=10em,
     command=\hskip0.5em\ldots\hskip0.5em\relax]

\setuplist  [section]
            [margin=10em, alternative=a]

\useURL[practex][http://www.tug.org/pracjourn/2006-2/schmitz/]

\setupitemize[1][autointro]

%% Frames and Backgrounds : <<<
\definetextbackground
    [EXAMPLE]
    [           mp=background:random,
          location=paragraph,
     rulethickness=1pt,
        framecolor=darkred,
             width=broad,
        leftoffset=1em,
       rightoffset=1em,
            before={\testpage[2]\blank[big]},
             after={\blank[big]}
%           before={\testpage[3]\blank[3*big]},
%            after={\blank[3*big]}
    ]

\startuseMPgraphic{background:random}
   path p;
   for i = 1 upto nofmultipars :
    p = (multipars[i]
     topenlarged 8pt
     bottomenlarged 8pt) randomized 4pt ;
   fill p withcolor lightgray ;
   draw p withcolor \MPvar{linecolor}
    withpen pencircle scaled \MPvar{linewidth};
   endfor;
\stopuseMPgraphic

\defineframedtext
    [EXAMPLEframe]
    [rulethickness=1pt,
        framecolor=darkred,
             width=broad,
        background=color,
   backgroundcolor=gray,
    ]

\defineoverlay[randomframe]
              [\useMPgraphic{background:random:frame}]

\startuseMPgraphic{background:random:frame}
   path p;
    p = (OverlayBox 
     topenlarged 10pt
     bottomenlarged 10pt) randomized 4pt ;
   fill p withcolor lightgray ;
   draw p withcolor \MPvar{linecolor}
    withpen pencircle scaled \MPvar{linewidth};
   endfor;
\stopuseMPgraphic
\setupexternalfigures[location={local,global,default}] 

%% >>>
%% Interface <<<
\definecolor[colorprettyfour][orange]
\usemodule[int-load]
\loadsetups[cont-en.xml]
\loadsetups[t-simpleslides.xml]
\definetextbackground
    [setuptext]
    [           mp=background:random,
          location=paragraph,
     rulethickness=1pt,
        framecolor=darkgreen,
             width=broad,
        leftoffset=1em,
       rightoffset=1em,
             align=right,
            before={\testpage[3]\blank[2*big]},
             after={\endgraf\blank[big]}
%           before={\testpage[3]\blank[3*big]},
%            after={\blank[3*big]}
    ]

%% There gotta be a better way to configure this!

\unprotected\def\showSETUPrecord
  {\getvalue{\e!start setuptext}
     \tttf
     \nohyphens
     \veryraggedright
     \startXMLmapping [one]
       \doglobal\newcounter\currentSETUPargument
       \global\let\maximumSETUPargument\currentSETUPargument
       \bgroup
         \doif{\XMLpar{cd:command}{generated}{}}{yes}{\ttsl}%
         \doifelseXMLop{type}{environment}
           {\tex{\e!start}}{\startcolor[colorprettytwo]\tex{}}\ignorespaces
           \XMLflush{cd:sequence}\stopcolor\ignorespaces
       \egroup
       \doifelseXMLempty{cd:arguments}
         {}
         {\bgroup
            \setbox0=\hbox{\XMLflush{cd:arguments}}%
            \global\let\maximumSETUPargument\currentSETUPargument
            \doglobal\newcounter\currentSETUPargument
            \ignorespaces\XMLflush{cd:arguments}%
            \doif{\XMLpar{cd:command}{type}{}}{environment}
              {\hskip.5em\unknown\hskip.5em
               \doif{\XMLpar{cd:command}{generated}{}}{yes}{\ttsl}%
               \tex{\e!stop}\ignorespaces\XMLflush{cd:sequence}}%
            \endgraf
          \egroup
         %\bgroup
         %  \tx
         %  \doif{\XMLpar{cd:command}{interactive}{}}{yes}      {\quad INTERACTIVE}%
         %  \doif{\XMLpar{cd:command}{interactive}{}}{exclusive}{\quad INTERACTIVE ONLY}%
         %\egroup
        \startXMLmapping [two]
          \bgroup
            \doglobal\newcounter\currentSETUPargument
            \blank[\v!line]
            %\switchtobodyfont[small] % kan sneller
            \ignorespaces\XMLflush{cd:arguments}\endgraf
            %\endgraf
          \egroup
        \stopXMLmapping}
     \stopXMLmapping
   \getvalue{\e!stop setuptext}}

\def\showSETUPnumber
  {\doglobal\increment\currentSETUPargument
   \hbox to 2em
     {\startcolor[blue]
      \ifcase\maximumSETUPargument\relax
        \or*\else\currentSETUPargument
      \fi\stopcolor
      \hss}}

\def\showSETUPassignment {\showSETUP
  {{\colorprettythree[}.\lower.5ex\hbox{=}.{\colorprettythree]}}
  {{\colorprettythree[}..,.\lower.5ex\hbox{=}.,..{\colorprettythree]}}}

\def\showSETUPkeyword {\showSETUP
  {\colorprettythree{[}...{\colorprettythree]}}
  {\colorprettythree{[}...,...{\colorprettythree]}}}

\def\showSETUPargument {\showSETUP
  {{\colorprettyone\leftargument}..{\colorprettyone\rightargument}}
  {{\colorprettyone\leftargument}..,...,..{\colorprettyone\rightargument}}}

\def\showSETUPdisplaymath {\showSETUP
  {\$\$...\$\$}
  {\$\$...\$\$}}

\def\showSETUPindex {\showSETUP
  {{\colorprettyone\leftargument}...{\colorprettyone\rightargument}}
  {{\colorprettyone\leftargument}..+...+..{\colorprettyone\rightargument}}}

\def\showSETUPmath {\showSETUP
  {\$...\$}
  {\$...\$}}

\def\showSETUPnothing {\showSETUP
  {...}
  {}}

\def\showSETUPfile {\showSETUP
  {~...~}
  {}}

\def\showSETUPposition {\showSETUP
  {(...)}
  {(...,...)}}

\def\showSETUPreference {\showSETUP
  {[...]}
  {[...,...]}}

\def\showSETUPcsname {\showSETUP
  {{\c!setup!command!{}}}
  {}}

\def\showSETUPdestination {\showSETUP
  {[{\colorprettyone\leftargument}..[ref]{\colorprettyone\rightargument}]}
  {[..,{\colorprettyone\leftargument}..[ref,..]{\colorprettyone\rightargument},..]}}

\def\showSETUPtriplet {\showSETUP
  {[x:y:z=]}
  {[x:y:z=,..]}}

\def\showSETUPword {\showSETUP
  {{\colorprettyone\leftargument}...{\colorprettyone\rightargument}}
  {{\colorprettyone\leftargument}.. ... ..{\colorprettyone\rightargument}}}

\def\showSETUPcontent {\showSETUP
  {{\colorprettyone\leftargument}...{\colorprettyone\rightargument}}
  {{\colorprettyone\leftargument}.. ... ..{\colorprettyone\rightargument}}}

%% >>>

\def\ShowStyle#1%
  {\blank[big]
   \midaligned{\startcombination[2*2]
     {\externalfigure[styles/#1][page=1,width=0.55\textwidth]}
     {Title Page}
     {\externalfigure[styles/#1][page=2,width=0.55\textwidth]}
     {Normal Slide}
     {\externalfigure[styles/#1][page=3,width=0.55\textwidth]}
     {Horizontal Picture}
     {\externalfigure[styles/#1][page=10,width=0.55\textwidth]}
     {Vertical Picture}
  \stopcombination}}

\starttext

\title
  Simple Slides \\
  A \CONTEXT\ presentation module 

\startEXAMPLE
\placelist[section]
\stopEXAMPLE

\section Introduction

This module provides an easy|-|to|-|use interface
for creating simple slides/presentations in \CONTEXT. 
The salient feature of this module are:
\startitemize
  \item The module is meant for presentations which will be shown on a
    digital projector. They have no interactive elements (such as buttons or
    hyperlinks) and no navigational tools (such as table of contents).
  \item The module comes with several predefined styles; these styles are sober
    in appearance and meant for academic presentations. It also provides some
    macros to help in presenting slides with both pictures and text.
  \item Most styles allow for some degree of user|-|reconfigurability. Designing
    a new style is also easy.
\stopitemize

This module provides a simple structure that will be suitable for beginning or
intermediate users of \CONTEXT, or someone who does not want to spend too much
time playing around with different configuration options for \CONTEXT. As such
it focusses on different users than Hans's presentation modules which have larger
and fancier features. This module also offers much less features than the
\LATEX\ \filename{beamer} package. Its main strength is its ease of use; you
should be able to write your first presentation after spending five minutes with
this manual.

\section A bit of history

The idea of a module suitable for simple presentations took shape when Thomas
started using \CONTEXT\ for preparing his course presentations. \CONTEXT\ comes
with a bunch of modules for presentations (the files \filename{s-pre-??.tex} in
\filename{$TEXMF/tex/context/base}) which are written by Hans Hagen. Hans
usually creates a new presentation style whenever he gives a talk about
\CONTEXT. As such, his presentation styles highlight the fancy and bleeding edge
features of \CONTEXT, and are not the most suitable starting point for academic
presentations. 

\CONTEXT\ does make creating your own presentation style relatively easy. So
Thomas wrote some presentation related macros (see the Prac\TEX\ article
{\tt\from[practex]}). With time, Thomas extended these macros into a collection
of styles providing different visual effects, and later collected all of them in
the \filename{taspresent} module. He gave a talk about the \filename{taspresent}
module in in the second \CONTEXT\ user meeting at Bohinj, and in the ensuing
discussions, Aditya and Thomas decided to modularize and
\quotation{\CONTEXT{ize}} some of the internals of the module, giving rise to
the current module. 

\section{Installation}

The module is installed in the usual way: simply unzip the archive
\filename{t-simpleslides-<date>.zip} into one of your \filename{$TEXMF} trees,
and from a terminal run \command{mktexlsr} (for \MKII) and \command{luatools
--generate} (for \MKIV). To verify that everything was installed correctly, from
a terminal run \command{kpsewhich t-simpleslides.tex} (for \MKII) and
\command{luatools t-simpleslides.tex} (for \MKIV); these commands should return
the complete path of the files that you just installed.

\subsubject A note about \TEX|-|engines

We have extensively tested this module with \PDFTEX\ and \LUATEX\
(that is, with \MKII\ and \MKIV). In spite of our best efforts, we have not been
able to get this module to work with \XETEX. If you are a \XETEX\ guru, and know
how to fix some of the errors with \XETEX, we will appreciate the help.

\section Quick start

First you must tell \CONTEXT\ that you want to use this module. To do this
simply write

\startEXAMPLE
\startTEX
\usemodule[simpleslides]
\stopTEX
\stopEXAMPLE

This sets the paper size and font sizes to values that are suitable for
presentations. Everything else is left like a default \CONTEXT\ document. The
module provides different styles that change the visual appearance of the
presentation. The following styles are presented:
\startitemize[columns, three]
  \item \options{BigNumber}
  \item \options{BottomSquares}
  \item \options{Boxed}
  \item \options{Ellipse}
  \item \options{Embossed}
  \item \options{Framed}
  \item \options{FramedTitle}
  \item \options{HorizontalStripes}
  \item \options{NarrowStripes}
  \item \options{RainbowStripe}
  \item \options{Rounded}
  \item \options{Shaded}
  \item \options{SideSquares}
  \item \options{Split}
  \item \options{Sunrise}
  \item \options{Swoosh}
  \item \options{ThickStripes}
\stopitemize

To use a style, say \options{BigNumber}, pass the \options{style=BigNumber}
option to the \filename{simpleslides} module as follows.

\startEXAMPLE
\startTEX
\usemodule[simpleslides]
          [style=BigNumber]
\stopTEX
\stopEXAMPLE

Some of the styles come have some variants, which can be chosen using
\options{color} and \options{bottom} keys. These are explained in \in
Section[sec:styles].

By default, the Latin Modern Sans font is used. The module also makes it easy to
use other fonts that come with a typical \TEX\ distribution. 
The following fonts are provided:
\startitemize[columns, three]
  \item \options{LatinModern}
  \item \options{LatinModernSans}
  \item \options{Bookman}
  \item \options{Chancery}
  \item \options{Gothic}
  \item \options{Helvetica}
  \item \options{Palatino}
  \item \options{Schoolbook}
  \item \options{Times}
\stopitemize

To choose a font, say \options{Helvetica}, pass \options{font=Helvetica} option
to the \filename{simpleslides} module as follows.
\startEXAMPLE
\startTEX
\usemodule[simpleslides]
          [style=BigNumber,
            font=Helvetica]
\stopTEX
\stopEXAMPLE

By default, the chosen font is used at 17pt size. The font size can be changed
using the \options{size} key.

More details about the fonts, including information on how to use your own fonts
is given in \in Section[sec:fonts]. 

The complete setup for using this module is
\setup{simpleslides}

\subsubject Structure of a presentation

The \filename{simpleslides} module has a very simple model of a presentation. A
presentation consists of a title followed by a series of slides; the module
provides macros to help create a presentation title page and slides.

\subsubject Presentation title page

A presentation title page displays the title of the presentation, the names of
the authors, and the date.
%% TODO: Also add institution and detail.
These can be specified using \typeTEX{\setupTitle} as follows:
\startEXAMPLE
\startTEX
\setupTitle
  [ title={Title of the presentation},
   author={Name of authors},
     date={Date of presentation},
  ]
\stopTEX
\stopEXAMPLE

The macro \typeTEX{\placeTitle} places the title page in the presentation. It is
possible to change the look of \typeTEX{\placeTitle} using some additional
arguments to \typeTEX{\setupTitle}. These are explained in \in
Section[sec:setuptitle].

\subsubject Presentation slide

The \filename{simpleslides} module provides a \typeTEX{\SlideTitle} macro, which
starts a new slide (basically a new page), and typesets its argument as the
title of the slide. It also takes care of increasing the page counters and
progress bars, and setting up the background. The content of the slides follows
after this command.

A slide is a normal \CONTEXT\ page, so you can use any commands and environments
that you want. Each presentation style sets up a style for itemizations, and
provides useful macros for placing pictures. These macros will be explained
later.

\adaptlayout[lines=+2]

\subsubject A minimal presentation

A minimal presentation is shown below. The result is shown in \in
Figure[fig:example].

\startEXAMPLE
\typefile[option=tex]{example.tex}
\stopEXAMPLE

\placefigure
  [top,bottom]
  [fig:example]
  {A minimal presentation}
  \startcombination[2*2]
    \startEXAMPLEframe[width=0.55\textwidth]
\startTEX
  \usemodule[simpleslides]
            [style=BigNumber]
  \setupTitle[...]

  \starttext
  \SlideTitle{...}
  ...
  \SlideTitle{...}
  ...
  \stoptext
\stopTEX
    \stopEXAMPLEframe
    {A minimal example}
    {\externalfigure[example][page=1,width=0.55\textwidth]}{Title page}
    {\externalfigure[example][page=2,width=0.55\textwidth]}{First slide}
    {\externalfigure[example][page=3,width=0.55\textwidth]}{Second slide}
  \stopcombination



\section Placing pictures

If you want to place pictures in your slides, you can always use \CONTEXT's
\typeTEX{\externalfigure} macro. This module also provides a macro,
\typeTEX{\IncludePicture}, for preconfigured picture layouts. Two layouts are
provided:
\startitemize
  \item \options{horizontal}: the picture is placed under the tile of the slide,
    so that it fits in the available space.
  \item \options{vertical}: the slide is divided into two colums; the picture is
    placed on the left column and text is placed on the right column.
\stopitemize
These layouts are shown in \in Figure[fig:pictures].

\page

\placefigure
  [top,bottom]
  [fig:pictures]
  {Example of \options{horizontal} and \options{vertical} options for 
  \typeTEX{\IncludePicture} macro}
  \startcombination[2*2]
    \startEXAMPLEframe[width=0.55\textwidth]
\startTEX
  \usemodule[simpleslides]
            [...]
  \starttext
  ...
  \IncludePicture
      [horizontal]
      [cow]
      {A Dutch Cow}
  ...
  \stoptext
\stopTEX
    \stopEXAMPLEframe
    {A horizontal picture}
    {\externalfigure[styles/BigNumber-blue]
                    [page=3,width=0.55\textwidth]}{A horizontal picture}
    \startEXAMPLEframe[width=0.55\textwidth]
\startTEX
  \usemodule[simpleslides]
            [...]
  \starttext
  ...
  \IncludePicture
      [vertical]
      [mill]
      {The windmills are an example of a green energy source} 
  ...
  \stoptext
\stopTEX
    \stopEXAMPLEframe
    {A vertical picture}
    {\externalfigure[styles/BigNumber-blue]
                    [page=10,width=0.55\textwidth]}{A vertical picture}
\stopcombination




A horizontal picture is placed as follows.
\startEXAMPLE
\startTEX
\IncludePicture
  [horizontal]
  [filename]  % Name of the file that contains the picture
  {Title of the slide}
\stopTEX
\stopEXAMPLE

while a vertical picture is placed as follows.
\startEXAMPLE
\startTEX
\IncludePicture
  [vertical]
  [filename]  % Name of the file that contains the picture
  {Text that is placed on the right of the picture}
\stopTEX
\stopEXAMPLE

It is possible to change the height and width of the pictures, or 
highlight them with circles and arrows. These details can be found in \in
Section[sec:pictures]

\page

\section[sec:styles] Changing presentation styles

The \options{style} key to \typeTEX{\setupmodule[simpleslides]} determines the
look of the presentation. Some of the styles come with variants, that can be
chosen using \options{color} and \options{bottom} keys. The available styles are
shown below along with the details of their varients.

\subsubject{BigNumber: with \options{color=blue} (also accepts \options{color=red})}

This is a style with subdued and quiet colors; its characteristic feature is the
page number on the lower right border of the text area. This detail was inspired
by {\em split} style (\filename{s-pre-14}) by Hans.

\ShowStyle {BigNumber-blue}
\page

\subsubject{BottomSquares}

This minimalistic style is inspired by a presentation Taco gave at EuroTeX
2006.

\ShowStyle {BottomSquares}
\page

\subsubject{Boxed}

This style is inspired by the screen version of the Metafun manual. Watch
the small gray circles at the bottom!

\ShowStyle {Boxed}
\page

\subsubject{Ellipse}

This style is inspired by {\em funny} style (\filename{s-pre-03}) by Hans. 
The light red stripe marks the progress.

\ShowStyle {Ellipse}
\page

\subsubject{Embossed}

Spread the word, don't be shy! Show your pride in using \CONTEXT. The color
theme will probably look familiar; I copied it from the \filename{enattab}
manual. 

\ShowStyle {Embossed}

If you are shy, or narcissist, you can change the emblem by
\startEXAMPLE
\startTEX
\setuplabeltext  [simpleslidesemblem={I made this presentation}]
\stopTEX
\stopEXAMPLE

\page

\subsubject{Framed: with \options{bottom=square}}

This style was inspired by {\em green} style (\filename{s-pre-02}) by Hans. It
has a thick blue frame around the entire slide area and a thinner frame around
the text area. The style has two options for the bottom area:
\options{bottom=stripe} will display a shaded blue area which will grow with
each slide; \options{bottom=square} displays a row of blue squares at the bottom
which also measure the presentation's progress.  

\ShowStyle {Framed-square}
\page

\subsubject{Framed: with \options{bottom=stripe}}
\ShowStyle {Framed-stripe}
\page

\subsubject{FramedTitle}

This is a style with loud titles. Its characteristic feature is the {\em scratch
counter} at the bottom. These are derived from Section~7.2 of the Metafun
manual.

\ShowStyle {FramedTitle}
\page

\subsubject{HorizontalStripes: with \options{color=green} (also accepts
\options{color=blue} and \options{color=red})}

A sober style with an emphasis on horizontal lines, inspired by {\em Szeged}
theme in \LATEX's \filename{beamer} package.

\ShowStyle {HorizontalStripes-green}
\page

\subsubject{NarrowStripes: with \options{color=green} (also accepts
\options{color=blue} and \options{color=red})}

A very simple and sober style, with shaded narrow stripes. 

\ShowStyle {NarrowStripes-blue}
\page

\subsubject{RainbowStripe}

A colorful style for daring presenters. The black line which marks the
progress is reminiscent of absorption lines in star spectra, so this style
may be apt for astrophysical presentations?

\ShowStyle {RainbowStripe}
\page

\subsubject{Rounded}

This style has cool colors and lots of white space; it is probably best suited
for presentations with relatively little text.

\ShowStyle {Rounded}
\page

\subsubject{Shaded: with \options{color=blue} (also accepts
\options{color=green} and \options{color=bluered})}

The only ornament to this style is the dark shaded background. It uses
\CONTEXT's {\tt interactionbar} mechanism to show the progress of the
presentation. It provides much space for text.

\ShowStyle {Shaded-blue}
\page

\subsubject{SideSquares}

This style is inspired by the colors and corporate look of Thomas's 
university. It is very sober and offers much space for text and
images. There is a rough progress meter built into the blue quadrangles. 

\ShowStyle {SideSquares}
\page

\subsubject{Split}

This style is inspired by the {\em Copenhagen} theme of the \LATEX's
\filename{beamer} package. The narrow blue and black stripes at the top and the
bottom of the slides display the date and slide number (top) and the title
and author of the presentation. 

\ShowStyle {Split}
\page

\subsubject{Sunrise}

This style is inspired by the {\em husky} theme of the \LATEX's
\filename{powerdot} package.

\ShowStyle {Sunrise}
\page

\subsubject{Swoosh}

Take a break from the right angles and straight lines. Use swooshy curves. This
style also has a fancy page counter at the bottom.

\ShowStyle {Swoosh}
\page

\subsubject{ThickStripes}

This theme is inspired by the {\em Berkeley} style of the \LATEX's
\filename{beamer} package. It has a stop watch at the bottom, which keeps tracks
of the number of slides.

\ShowStyle {ThickStripes}

\section[sec:fonts] Changing presentation fonts

The \options{font} and the \options{size} keys to
\typeTEX{\setupmodule[simpleslides]} determine the font and font size for the
main text of the presentation. The default font is Latin Modern Sans at 17pt. 

\startitemize
\item The \options{font} key can take the following values.

\starttabulate[|l|p|]
\NC \options{LatinModern}     \NC typesets in Latin Modern Serif     \NC \NR
\NC \options{LatinModernSans} \NC typesets in Latin Modern Sans      \NC \NR
\NC \options{Bookman}         \NC typesets in \TeX Gyre Bonum (a Bookman
    clone)       \NC \NR
\NC \options{Chancery}        \NC typesets in \TeX Gyre Chorus 
    \footnote{Please be aware that Chorus is a calligraphic font. It has no
    italic or bold.} (a Zapf Chancery clone)                        \NC \NR
\NC \options{Gothic}          \NC typesets in \TeX Gyre Adventor (a Gothic
    clone)   \NC \NR
\NC \options{Helvetica}       \NC typesets in \TeX Gyre Heros (a Helvetica
    clone)       \NC \NR
\NC \options{Palatino}        \NC typesets in \TeX Gyre Pagella (a Palatino
    clone)   \NC \NR
\NC \options{Schoolbook}      \NC typesets in \TeX Gyre Schola (a Schoolbook
    clone)     \NC \NR
\NC \options{Times}           \NC typesets in \TeX Gyre Termes (a Times clone)
\NC \NR
\stoptabulate

\item The \options{size} key can be any valid \TEX\ dimension.

\stopitemize

\subsubject Choosing your own font

If you want to set up your own font, pick any value for the \options{font} key
(or leave it empty). Use the \options{size} key to choose the font size. Then
{\em after} loading the module, choose any font using the normal \CONTEXT\
commands. Make sure to set the bodyfont at size \typeTEX{\NormalSize}. So, if
you have your own typescript for a font, your setup will look like this:

\startEXAMPLE
\startTEX
\usemodule[simpleslides][...]
....
\usetypescriptfile[type-myfont]    % The typescript for your font
\usetypescript[Mytypescript]       % As set in your typescript file
\setupbodyfont[Myfont,\NormalSize] % Note the \NormalSize here
\stopTEX
\stopEXAMPLE

Internally, the font size is stored in the macro \typeTEX{\NormalSize}. The main
text is set at size \typeTEX{\NormalSize}; the title page and the slide title
are set at size \typeTEX{\TitleSize}. The module uses some heuristics to select
a reasonable value of \typeTEX{\TitleSize}. If you do not like ths size of the
title page and slide titles, you can change the value to \typeTEX{\TitleSize} to
whatever you like.

\section {Details}
\subsection{Macros}

The module offers only a few macros to facilitate the preparation of
presentations. 

\subsection{\typeTEX{\setupTitle}}

Begin your presentation by setting the name of the author(s), the title,
and the date with this macro:

\setup{setupTitle}

If you don't set the date, it will default to \tex{currentdate}.

\subsection{\typeTEX{\MakeTitle}}

This macro will produce a title page with the author, title, and date of
the presentation; the look is of course determined by the style of your
presentation. 

\subsection{\typeTEX{\IncludePicture}}

This macro facilitates the placement of pictures. It takes four arguments:

\setup{IncludePicture}

The first argument decides whether the picture will be placed in a horizontal
or vertical arrangement; for examples, refer to fig. \in[horizontal] on
p.~\at[horizontal] and fig. \in[vertical] on p.~\at[vertical]. The second
argument is the filename of the picture you want to include. The fourth
argument (in braces) is the text accompanying the picture, which will be placed
either in a \tex{SlideTitle} environment (for horizontal pictures) or opposite
the picture, centered horizontally and vertically, for vertical pictures.

\setup{setupPicture}

The third argument is the most complex; it does the setup for the included
picture. Here is a brief explanation of what the different parameters do:

\starttabulate[|lw(.22\textwidth)|p|]
\NC {\tt width} and {\tt height} 
\NC Unsurprisingly, these set the height or width for the picture. Normally, the
    module will automatically fit your pictures to fill the available space, so
    you only need to set one of these values if you want to override this
    mechanism. 
\NC\NR
\NC {\tt highlight} 
\NC This boolean key decides whether elements highlighting certain areas of the
    picture will be placed on the slide. 
\NC \NR
\NC {\tt alternative} 
\NC Three different forms of highlighting are available: {\tt ellipse} places a
    circle on top of the picture (see fig.~\in[circle] on p.~\at[circle]), {\tt
    arrow} an arrow (see fig.~\in[arrow] on p.~\at[arrow]), and {\tt focus} a gray overlay in which a circle is
    transparent (see fig.~\in[gray] on p.~\at[gray]). 
\NC\NR
\NC {\tt color} 
\NC The color of the circle or arrow on the picture. You can either use a
    predefined color (see chapter 6.2 of the \CONTEXT\ manual) or make one up
    yourself in the form {\tt (r=.5,g=.3,b=.9)} 
\NC \NR
\NC {\tt rulethickness}
\NC The thickness of the highlighting arrows or ellipses.
\NC \NR
\NC {\tt x} and {\tt y} 
\NC These variables determine the horizontal and vertical positioning of the
    highlighting elements; they point to the center of the circle or to the tip
    of the arrow.% If you want more than one circle or arrow, simply set values 	   for x2, y2; x3, y3, etc.; up to five circles/arrows are defined.
\NC \NR
\NC {\tt xscale} and {\tt yscale} 
\NC The scale of the ellipse, calculated in relation to the size of the
    picture. If you want a circle instead of an ellipse, give just one of
    the numbers and set the other to {\tt couple}. 
\NC \NR
\NC {\tt length}
\NC The length of the arrow (dimension).
\NC \NR
\NC {\tt direction} 
\NC The rotation of the ellipse resp. the direction of the arrow, given in degrees. 
\NC \NR
\NC {\tt opacity} 
\NC  Sets the opacity of the gray layer (with {\tt 0} = transparent and {\tt
    100} = completely opaque).
\NC \NR
\NC {\tt shadow}
\NC Highlighting ellipses and arrows can have shadows attached, which will
    make them stand out more on your slides. By default, shadows are
    disabled. This key determines the place where this shadows will appear;
    possible values are {\tt topleft}, {\tt topright}, {\tt bottomleft},
    and {\tt bottomright}. Simply setting this key to {\tt yes} puts the
    shadow {\tt bottomright}. 
\NC \NR
\NC {\tt grid} 
\NC Boolean. If set to {\tt yes}, it places a grid on top of the
    picture. The grid divides the height and width of the picture into 10
    sections; this
    is helpful for determining the exact position where you want to place
    circles and arrows; see fig.~\in[grid] on p.~\at[grid]. 
\NC \NR
\NC {\tt subgrid}
\NC Boolean. Use this to subdivide the grid to get more fine-grained
    control of the position of your highlighting elements.
\NC \NR
\NC {\tt steps}
\NC The number of subdivisions of the {\tt subgrid},
\NC \NR
\NC {\tt gridcolor} 
\NC The color of the grid; default is {\tt black}. 
\NC \NR
\stoptabulate

\placefigure
    [here]
    [horizontal]
    {Horizontal picture with the \filename{NarrowStripes} style}
    {\externalfigure[styles/NarrowStripes-blue][page=3,width=.66\textwidth]} 

\placefigure
    [here]
    [vertical]
    {Vertical picture with the \filename{DoubleFrame} style}
    {\externalfigure[styles/RainbowStripe][page=10,width=.66\textwidth]} 

\placefigure
    [here]
    [circle]
    {Orange circle as highlighting mechanism, {\tt Split} style}
    {\externalfigure[styles/Split][page=7,width=.66\textwidth]} 

\placefigure
    [here]
    [arrow]
    {Orange arrow as highlighting mechanism, {\tt Split} style}
    {\externalfigure[styles/NarrowStripes-red][page=8,width=.66\textwidth]} 

\placefigure
    [here]
    [gray]
    {Focus as highlighting mechanism, {\tt HorizontalStripes} style}
    {\externalfigure[styles/HorizontalStripes-blue][page=9,width=.66\textwidth]} 


\stoptext

% vim: foldmethod=marker foldmarker=<<<,>>>
