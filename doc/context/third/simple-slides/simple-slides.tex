\setupcolors[state=start]

\usemodule[int-load]

\usemodule[abr-01]

\setupcolors[state=start]

\loadsetups[cont-en.xml]
\loadsetups[simpleslides.xml]

\definetype[typeTEX][option=tex,style=type]

\setuphead[title][alternative=middle] % Will make it fancy later

\useURL[practex][http://www.tug.org/pracjourn/2006-2/schmitz/]

\setupexternalfigures[location={local,global,default}] 

\starttext

\title
  Simple Slides \\
  A \CONTEXT\ module for making slides 

\section Introduction

This module provides an easy|-|to|-|use interface
for creating simple slides|/|presentations in \CONTEXT. 
The salient feature of this module are:
\startitemize[intro]
  \item The module is meant for presentations which will be shown on a
    digital projector. They have no interactive elements (such as buttons or
    hyperlinks) and no navigational tools (such as table of contents).
  \item The module comes with several predefined styles; these styles are sober
    in appearance and meant for academic presentations. It also provides some
    macros to help in presenting slides with both pictures and text.
  \item Most styles allow for some degree of user|-|reconfigurability. Designing
    a new style is also easy.
\stopitemize

This module provides a simple structure that will be suitable for beginning or
intermediate users of \CONTEXT, or someone who does not want to spend too much
time playing around with different configuration options for \CONTEXT. As such
it focusses on different users than Hans's presentation modules which have larger
and fancier features. This module also offers much less features than the
\LATEX\ \filename{beamer} package. Its main strength is its ease of use; you
should be able to write your first presentation after spending five minutes with
this manual.

\section A bit of history

The idea of a module suitable for simple presentations took shape when Thomas
started using \CONTEXT\ for preparing his course presentations. \CONTEXT\ comes
with a bunch of modules for presentations (the files \filename{s-pre-??.tex} in
\filename{$TEXMF/tex/context/base}) which are written by Hans Hagen. Hans
usually creates a new presentation style whenever he gives a talk about
\CONTEXT. As such, his presentation styles highlight the fancy and bleeding edge
features of \CONTEXT, and are not the most suitable starting point for academic
presentations. 

\CONTEXT\ does make creating your own presentation style relatively easy. So
Thomas wrote some presentation related macros (see the Prac\TEX\ article
{\tt\from[practex]}). With time, Thomas extended these macros into a collection
of styles providing different visual effects, and later collected all of them in
the \filename{taspresent} module. He gave a talk about the \filename{taspresent}
module in in the second \CONTEXT\ user meeting at Bohinj, and in the ensuing
discussions, Aditya and Thomas decided to modularize and
\quotation{\CONTEXT{ize}} some of the internals of the module, giving rise to
the current module. 

\section{Installation}

The module is installed in the usual way: simply unzip the archive
\filename{t-simpleslides-<data>.zip} into one of your \filename{$TEXMF} tree,
and from a terminal run \type{mktexlsr} (for \MKII) and \type{luatools
--generate} (for \MKIV). To verify that everything was installed correctly, from
a terminal run \type{kpsewhich t-simpleslides.tex} (for \MKII) and
\type{luatools t-simpleslides.tex} (for \MKIV); these commands should return the
complete path of the files that you just installed.

\section{Quick start}

Here's an example of the most basic way of using the module:

\startTEX
\usemodule[simpleslides]
	  [style=Swoosh,
	   font=Palatino,
	   size=17pt]
 
\setupTitle
       [title={Title of the presentation},
       author={Author of the presentation},
         date={date of the presentation}]
 
\starttext
\placeTitle
 
\SlideTitle{First Slide}
 
Your text goes here.

\SlideTitle{Second Slide}

More text.
 
\stoptext
\stopTEX

The command \typeTEX{\SlideTitle} begins a new slide, typesets its argument as
the title of this slide, takes care of increasing counters and progress
bars, and sets up backgrounds; the content of your slides follows after
this command. Your slide is just a normal \CONTEXT\ environment, so you can
use any commands and environments you want. If you want to have slides with
illustrations, there are two ways to place pictures: in the
\quotation{horizontal} structure, the picture is placed under the title of
the slide; in the \quotation{vertical} structure, the picture is placed on
the left-hand side of the slide and the text opposite it. Pictures are
placed with this command:

\startTEX
\IncludePicture[horizontal]
               [NAME]
	       [width=\textwidth]
	       {This text will be the title of the slide}

\IncludePicture[vertical]
               [NAME]
	       [width=\NormalWidth]
	       {This text will be placed opposite the picture}
\stopTEX

The macros \tex{NormalWidth} and \tex{NormalHeight} make sure that
your vertical picture fills the available space.

\section{Using the Module}

\subsection{The setup command}

To use the module, you put this line into your source file:

\setup{simpleslides}

The second argument takes a list of {\tt key=value} assignments which you
can also set with this command:

\setup{setupsimpleslides}

The values for the different keys will be explained in the following
sections. 

\subsubsection{The {\tt style} Key}

The {\tt style} key determines the look of your presentation; it defines
the background, borders, elements of the slides, etc.  There are nineteen
predefined options for this key; the appendix shows what they look like. 

\subsubsection{The {\tt size} Key}

This selects the font size for the main text. The value has to be a
dimension; it is stored in the macro \tex{NormalSize}. The \tex{TitleSize}
is calculated according to the main text size.

\subsubsection{The {\tt font} Key}

You can use any fonts and typescripts with the module. In order to
facilitate choosing a font, especially for beginners, the module provides
shortcuts for using the \TeX Gyre font collection and for Latin
Modern. These predefined fonts should work on any \TeX\ installation that
has the complete \TeX Gyre collection; this is the case for the \CONTEXT\
minimals and for \TeX Live since 2008. They are selected by setting the
{\tt font} key.

\starttabulate[|l|p|]
\NC {\tt LatinModern}     \NC typesets in LatinModern Serif     \NC \NR
\NC {\tt LatinModernSans} \NC typesets in LatinModern Sans      \NC \NR
\NC {\tt Bookman}         \NC typesets in \TeX Gyre Bonum       \NC \NR
\NC {\tt Chancery}        \NC typesets in \TeX Gyre Chorus
    \footnote{Please be aware that Chorus is a calligraphic font. It has no
    italic or bold.}                                            \NC \NR
\NC {\tt Gothic}          \NC typesets in \TeX Gyre Adventor    \NC \NR
\NC {\tt Helvetica}       \NC typesets in \TeX Gyre Heros       \NC \NR
\NC {\tt Palatino}        \NC typesets in \TeX Gyre Pagella     \NC \NR
\NC {\tt Schoolbook}      \NC typesets in \TeX Gyre Schola      \NC \NR
\NC {\tt Times}           \NC typesets in \TeX Gyre Termes      \NC \NR
\stoptabulate
 
If you set your own font, please remember to select the bodyfont at
\typeTEX{\NormalSize} and to give your setup commands {\em after} loading
the module (or \TeX\ will not know what \typeTEX{\NormalSize} means and
complain about an \quotation{undefined control sequence}). So if you have
your own typescript, your setup should look like this:

\startTEX
\usetypescriptfile[type-myfile]
\usetypescript[Mytypescript]
\setupbodyfont[Myfont,\NormalSize]
\stopTEX

\subsection{Macros}

The module offers only a few macros to facilitate the preparation of
presentations. 

\subsection{\typeTEX{\setupTitle}}

Begin your presentation by setting the name of the author(s), the title,
and the date with this macro:

\setup{setupTitle}

If you don't set the date, it will default to \tex{currentdate}.

\subsection{\typeTEX{\MakeTitle}}

This macro will produce a title page with the author, title, and date of
the presentation; the look is of course determined by the style of your
presentation. 

\subsection{\typeTEX{\IncludePicture}}

This macro facilitates the placement of pictures. It takes four arguments:

\setup{IncludePicture}

The first argument decides whether the picture will be placed in a horizontal
or vertical arrangement; for examples, refer to fig. \in[horizontal] on
p.~\at[horizontal] and fig. \in[vertical] on p.~\at[vertical]. The second
argument is the filename of the picture you want to include. The fourth
argument (in braces) is the text accompanying the picture, which will be placed
either in a \tex{SlideTitle} environment (for horizontal pictures) or opposite
the picture, centered horizontally and vertically, for vertical pictures.

\setup{setupPicture}

The third argument is the most complex; it does the setup for the included
picture. Here is a brief explanation of what the different parameters do:

\starttabulate[|lw(.22\textwidth)|p|]
\NC {\tt width} and {\tt height} 
\NC Unsurprisingly, these set the height or width for the picture. Normally, the
    module will automatically fit your pictures to fill the available space, so
    you only need to set one of these values if you want to override this
    mechanism. 
\NC\NR
\NC {\tt highlight} 
\NC This boolean key decides whether elements highlighting certain areas of the
    picture will be placed on the slide. 
\NC \NR
\NC {\tt alternative} 
\NC Three different forms of highlighting are available: {\tt ellipse} places a
    circle on top of the picture (see fig.~\in[circle] on p.~\at[circle]), {\tt
    arrow} an arrow (see fig.~\in[arrow] on p.~\at[arrow]), and {\tt focus} a gray overlay in which a circle is
    transparent (see fig.~\in[gray] on p.~\at[gray]). 
\NC\NR
\NC {\tt color} 
\NC The color of the circle or arrow on the picture. You can either use a
    predefined color (see chapter 6.2 of the \CONTEXT\ manual) or make one up
    yourself in the form {\tt (r=.5,g=.3,b=.9)} 
\NC \NR
\NC {\tt rulethickness}
\NC The thickness of the highlighting arrows or ellipses.
\NC \NR
\NC {\tt x} and {\tt y} 
\NC These variables determine the horizontal and vertical positioning of the
    highlighting elements; they point to the center of the circle or to the tip
    of the arrow.% If you want more than one circle or arrow, simply set values 	   for x2, y2; x3, y3, etc.; up to five circles/arrows are defined.
\NC \NR
\NC {\tt xscale} and {\tt yscale} 
\NC The scale of the ellipse, calculated in relation to the size of the
    picture. If you want a circle instead of an ellipse, give just one of
    the numbers and set the other to {\tt couple}. 
\NC \NR
\NC {\tt length}
\NC The length of the arrow (dimension).
\NC \NR
\NC {\tt direction} 
\NC The rotation of the ellipse resp. the direction of the arrow, given in degrees. 
\NC \NR
\NC {\tt opacity} 
\NC  Sets the opacity of the gray layer (with {\tt 0} = transparent and {\tt
    100} = completely opaque).
\NC \NR
\NC {\tt shadow}
\NC Highlighting ellipses and arrows can have shadows attached, which will
    make them stand out more on your slides. By default, shadows are
    disabled. This key determines the place where this shadows will appear;
    possible values are {\tt topleft}, {\tt topright}, {\tt bottomleft},
    and {\tt bottomright}. Simply setting this key to {\tt yes} puts the
    shadow {\tt bottomright}. 
\NC \NR
\NC {\tt grid} 
\NC Boolean. If set to {\tt yes}, it places a grid on top of the
    picture. The grid divides the height and width of the picture into 10
    sections; this
    is helpful for determining the exact position where you want to place
    circles and arrows; see fig.~\in[grid] on p.~\at[grid]. 
\NC \NR
\NC {\tt subgrid}
\NC Boolean. Use this to subdivide the grid to get more fine-grained
    control of the position of your highlighting elements.
\NC \NR
\NC {\tt steps}
\NC The number of subdivisions of the {\tt subgrid},
\NC \NR
\NC {\tt gridcolor} 
\NC The color of the grid; default is {\tt black}. 
\NC \NR
\stoptabulate

\placefigure
    [here]
    [horizontal]
    {Horizontal picture with the \filename{NarrowStripes} style}
    {\externalfigure[styles/NarrowStripes-blue][page=3,width=.66\textwidth]} 

\placefigure
    [here]
    [vertical]
    {Vertical picture with the \filename{DoubleFrame} style}
    {\externalfigure[styles/RainbowStripe][page=10,width=.66\textwidth]} 

\placefigure
    [here]
    [circle]
    {Orange circle as highlighting mechanism, {\tt Split} style}
    {\externalfigure[styles/Split][page=7,width=.66\textwidth]} 

\placefigure
    [here]
    [arrow]
    {Orange arrow as highlighting mechanism, {\tt Split} style}
    {\externalfigure[styles/NarrowStripes-red][page=8,width=.66\textwidth]} 

\placefigure
    [here]
    [gray]
    {Focus as highlighting mechanism, {\tt HorizontalStripes} style}
    {\externalfigure[styles/HorizontalStripes-blue][page=9,width=.66\textwidth]} 


\section{Simpleslides and the \TeX|-|engines}

We have not been able to test all available styles, fonts, colors,
etc. with all engines, and we are aware that there are some problems
especially with the \XeTeX\ engine (which neither of us uses a lot), but in
principle, the module should work with traditional pdf\TeX\ (\type{mkii}),
with \XeTeX\ and with lua\TeX\ (\type{mkiv}). 

\page

\subject{Appendix: examples of the different styles}

\startcombination[2*3]
  {\externalfigure[styles/BigNumber-blue][page=2,width=.5\textwidth]}{The {\tt BigNumber} style, {\tt color} blue}
  {\externalfigure[styles/BottomSquares][page=2,width=.5\textwidth]}{The {\tt BottomSquares} style}
  {\externalfigure[styles/Boxed][page=2,width=.5\textwidth]}{The {\tt Boxed} style}
  {\externalfigure[styles/Ellipse][page=2,width=.5\textwidth]}{The {\tt Ellipse} style}
  {\externalfigure[styles/Embossed][page=2,width=.5\textwidth]}{The {\tt Embossed} style}
  {\externalfigure[styles/Framed-stripes][page=2,width=.5\textwidth]}{The {\tt Framed} style}
\stopcombination

\startcombination[2*3]
  {\externalfigure[styles/FramedTitle][page=2,width=.5\textwidth]}{The {\tt FramedTitle} style}
  {\externalfigure[styles/FramedSquare][page=2,width=.5\textwidth]}{The {\tt FramedSquare} style}
  {\externalfigure[styles/HorizontalStripes-blue][page=2,width=.5\textwidth]}{The {\tt HorizontalStripes} style}
  {\externalfigure[styles/NarrowStripes-green][page=2,width=.5\textwidth]}{The {\tt NarrowStripes} style}
  {\externalfigure[styles/RainbowStripe][page=2,width=.5\textwidth]}{The {\tt RainbowStripe} style}
  {\externalfigure[styles/Rounded][page=2,width=.5\textwidth]}{The {\tt Rounded} style}
\stopcombination

\startcombination[2*3]
  {\externalfigure[styles/Shaded-blue][page=2,width=.5\textwidth]}{The {\tt Shaded} style}
  {\externalfigure[styles/SideSquares][page=2,width=.5\textwidth]}{The {\tt SideSquares} style}
  {\externalfigure[styles/Split][page=2,width=.5\textwidth]}{The {\tt Split} style}
  {\externalfigure[styles/Sunrise][page=2,width=.5\textwidth]}{The {\tt Sunrise} style}
  {\externalfigure[styles/Swoosh][page=2,width=.5\textwidth]}{The {\tt Swoosh} style}
  {\externalfigure[styles/ThickStripes][page=2,width=.5\textwidth]}{The {\tt ThickStripes} style}
\stopcombination

\stoptext
